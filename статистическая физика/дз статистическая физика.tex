\documentclass[a4paper,12pt]{article} % тип документа


\usepackage[T2A]{fontenc} % кодировка
\usepackage[utf8]{inputenc} % кодировка исходного текста
\usepackage[english,russian]{babel} % локализация и переносы

% colors
\usepackage[dvipsnames,table,xcdraw]{xcolor}          
\definecolor{light-blue}{rgb}{0.8,0.85,1}


%symbols
\usepackage{upgreek}
\usepackage{amsmath,amsfonts,amssymb,amsthm,mathtools} %math
% теоремы
\theoremstyle{plain}
\newtheorem{definition}{Определение}[section] 
\newtheorem{theorem}{Теорема}
\newtheorem{example}{Пример}
\numberwithin{equation}{section}



% гиперссылки:
\usepackage{hyperref}
\definecolor{darkblue}{HTML}{0000A0}
\definecolor{linkcolor}{HTML}{0000FF}
\hypersetup{pdfstartview=FitH, citecolor=linkcolor, linkcolor=darkblue,urlcolor=red, colorlinks=true}


% графика:
\usepackage{graphics}
\graphicspath{{pic/}}
\DeclareGraphicsExtensions{.pdf,.png,.jpg, .eps}
\usepackage{caption} % а что без этого летит? (забыл)
\usepackage[section,above,below]{placeins} % управление плавающими объектами (?)
\usepackage{floatflt}
\usepackage{framed}



% работа с таблицами (?)
\usepackage{multirow}
\newcommand{\specialcell}[2][c]{%
	\begin{tabular}[#1]{@{}c@{}}#2\end{tabular}} % перенос строки в ячейке таблицы при пакете multirow


\newcommand{\comment}[1]{} % for multiline comments

% defining red box
\newsavebox{\selvestebox}
\newenvironment{colbox}[1]
{\newcommand\colboxcolor{#1}%
	\begin{lrbox}{\selvestebox}%
		\begin{minipage}{\dimexpr\columnwidth-2\fboxsep\relax}}
		{\end{minipage}\end{lrbox}%
	\begin{center}
		\colorbox[HTML]{\colboxcolor}{\usebox{\selvestebox}}
\end{center}}

% предметный указатель и библиография
\usepackage{makeidx}
\makeindex
\usepackage[nottoc]{tocbibind}

% разметка и стиль (???)
\usepackage[left=2cm, right=2cm, top=2cm, bottom=2cm]{geometry}
\usepackage{fancyhdr}
\pagestyle{fancy}
\fancyhead[L]{\rightmark}
%\lhead{ краткое название}
\chead{}
\rhead{\thepage}
\cfoot{} % get rid of the page number 
\renewcommand{\headrulewidth}{1pt}
\renewcommand{\footrulewidth}{0pt}
\usepackage{indentfirst}

\usepackage{framed}
\usepackage{fancyvrb} %for fraim aroun verbatim
 % основная шапка


\author{Юрий Голубев\\ yura.winter@gmail.com }
\title{статистическая физика}
\date{\today}


\begin{document}
\maketitle

\begin{abstract}
статистическая физика
\end{abstract}
%%%%%%%%%%%%%%%%%%%%%%%%%%%%%%%%%%%%%%%%%%%%%%%%%%%%%%%%%%%%%%	
\tableofcontents

\section*{Предисловие}
\addcontentsline{toc}{section}{Предисловие}




\clearpage
\part{упражнения}

\section{модели статфиза}

(потом назову главы по типам)

\begin{task}textbf{1}

$ N $ молекул идеального газа в объеме $ V $. Определить вероятность того, что в объеме $v < V$ находится $ n $ молекул.

Получить приближенное выражение, когда  $v \le V $. 
Найти среднее число частиц n в объеме v, его среднюю абсолютную и относительную флуктуации. Найти вид распределения в случае v V, n  1. 


\end{task}


\begin{task}

2
Вычислить $ c_p-C_v$ в  переменных V, T и P, T. 

Определить $ c_p-C_v$ для больцмановского газа,
газа Ван-дер-Ваальса,фермии бозе-газа и черного излучения


\end{task}


\begin{task}

3. Вычислить число состояний одноатомного больцмановского газа


\end{task}


\begin{task}

4

Вычислить число состояний системы N независимых спинов 1/2.


\end{task}


\begin{task}

5. Вычислитьчислосостоянийсистемы N одинаковыхнезависимых осцилляторов



\end{task}


\begin{task}

6

Получить выражения для неравновесной энтропии ферми- и бозе-газов


\end{task}


\begin{task}
7

Вычислить основные термодинамические величины ферми- и

бозе-газов при $T=0$


\end{task}


\begin{task}

8

Из функционала Гинзбурга–Ландау получить выражение для плотности тока в магнитном поле, получить уравнение Лондонов и квантование магнитного потока в сверхпроводящем кольце.


\end{task}


\begin{task}

9
Вычислить среднее от произведения четырех ферми-операторов  $\left\langle\hat{a}_{k}^{+} \hat{a}_{p}^{+} \hat{a}_{u} \hat{a}_{v}\right\rangle,$ где $\langle\cdots\rangle-$ усреднение по состоянию невзаимодействующих частиц с заданной температурой и химпотенциалом.

\end{task}


\begin{task}

10

Записать оператор взаимодействия электронов с внешними электрическим и магнитным полями в представлении вторичного квантования.


\end{task}


\begin{task}

11

Вычислить  $\langle\exp (-i q \hat{x})\rangle, \text { гДе } \hat{x}$ оператор смещения одномерного гармонического осциллятора.




\end{task}


\begin{task}

12

Определить температурную зависимость среднеквадратичного смещения атомов от положения равновесия Rk Rp, где ··· обозначают усреднение по состоянию невзаимодействующих фононов с заданной температурой,  Rk –с мещениеатомав k-направлении. Объяснить происхождение нулевых колебаний.



\end{task}


\begin{task}

13

Используя результаты предыдущей задачи, вычислить среднее от произведения четырех операторов смещения, относяпихся к одной и той же ячейке: $\left\langle\hat{R}_{k} \hat{R}_{p} \hat{R}_{i} \hat{R}_{j}\right\rangle,$ где $\langle\cdots\rangle$ обозначают усреднение по состоянию невзаимодействуюших фононов с заданной температурой, $\hat{R}_{k}-$ смещение атома в $k$ -направлении $(k=x, y, z)$



\end{task}


\begin{task}

14

Для электронов, находящихся под поверхностью Ферми, произвести переход к дырочному представлению. Записать полный гамильтониан идеального ферми-газа, используя операторы рождения и уничтожения квазичастиц (электронов над поверхностью Ферми и дырок под поверхностью Ферми). Определить химический потенциал и энергетический спектр полученных квазичастиц. 



\end{task}


\begin{task}

15

Вычисляя первую поправку термодинамической теории возмущений, найти вклад прямого и обменного взаимодействия для ферми- и бозе-частиц. Сравнить результаты


\end{task}


\begin{task}

16




\end{task}


\begin{task}

В преобразовании Боголюбова для электронов получить при T  Tc связь операторов поглощения квазичастиц и поглощения голых электронов.


\end{task}




\clearpage
\part{задачи}

\section{модели статфиза}

(мб добавлю разбивку на разделы позже)

\begin{task}
(задача 1)

Показать, что замкнутая система из двух равновесных подсистем имеет максимальную энтропию, 
когда у подсистемы равны температура, давление и химические потенциалы.


Энтропию замкнутой системы, образованной из двух равновесных подсистем, определим как:
$$
S=S_{1}\left(E_{1}\right)+S_{2}\left(E_{2}\right)
$$


Такая энтропия максимальна, когда обе подсистемы имеют одинаковую температуру, химический потенциал и давление.
Действительно:
при постоянстве полной энергии $E=E_{1}+E_{2}$ будет выполняться:
{\Large $$
	\begin{array}{c}
	\frac{d S}{d E_{1}}=\frac{d S_{1}}{d E_{1}}+\frac{d S_{2}}{d E_{2}} \frac{d\left(E-E_{1}\right)}{d E_{1}}=\frac{1}{T_{1}}-\frac{1}{T_{2}}=0 \\
	\frac{d^{2} S}{d E^{2}}=-\left(\frac{1}{T^{2} C_{V}}\right)_{1}-\left(\frac{1}{T^{2} C_{V}}\right)_{2}<0
	\end{array}
	$$}
Поэтому в случае максимума энтропии температуры подсистем одинаковы.

Аналогично, варьируя энтропию системы по объему и числу част иц одной из подсистем при условии постоянства полного объема и полного числа частиц 
$\left(\right.$ Используя $\left.\frac{\partial S}{\partial V}=
\frac{\partial S}{\partial E} \frac{\partial E}{\partial V}=\frac{1}{T}(-p), \frac{\partial S}{\partial N}=
\frac{\partial S}{\partial E} \frac{\partial E}{\partial N}=
\frac{1}{T} \mu\right),$ находим, что Энтропия
максимальна, когда равны друг другу давления и химические потенциалы подсистем.



\end{task}


\begin{task}
(задача 2)

Найти кривую фазового равновесия газ-жидкость P(T).


\end{task}


\begin{task}

3

Определить энтропию газа $ N $ невзаимодействующих спинов  $\sigma= 1/2$ в магнитном поле при заданной энергии. 
Определить понятие температуры и показать, что она может быть отрицательной. 
Обсудить температурную зависимость теплоемкости. 
Сравнить с задачей о системе невзаимодействующих двухуровневых частиц







\end{task}


\begin{task}
4. Определить энтропию газа N невзаимодействующих осцилляторов при заданной энергии E. Получить связь между энергией и температурой T. Обсудить отличие температурного поведения теплоемкости от предыдущей задачи.



\end{task}


\begin{task}

5. Вычислить магнитную восприимчивость одноатомного парамагнитного газа (T) с моментом J.


\end{task}


\begin{task}

$6 .$ Вычислить для парамагнитного газа изменение температуры при адиабатическом изменении магнитного поля $(\partial T / \partial H)_{S},$ если его свободная энергия может быть представлена в виде: $F=F_{0}(T)-$ $-(1 / 2) \chi(T) H^{2}$


\end{task}


\begin{task}

7

$$
\begin{aligned}
&\text { 7. Найти флуктуации } \overline{\Delta E^{2}}, \overline{\Delta N^{2}}, \overline{\Delta S^{2}}, \overline{\Delta P^{2}}, \overline{\Delta S \Delta P}, \overline{\Delta V \Delta P} \text { , }\\
&\overline{\Delta S \Delta T}, \overline{\Delta T^{2}}, \overline{\Delta V^{2}}, \overline{\Delta T \Delta V}, \overline{\Delta T \Delta P}, \overline{\Delta S \Delta V}
\end{aligned}$$


\end{task}


\begin{task}

8 Вычислить для одноатомного и двухатомного больцмановских газов 

$ F, \mu, P, S, C,(\partial P)(\partial \rho)_{S}$




\end{task}


\begin{task}

Найти теплоемкость идеального газа без внутренних степеней свободы, помещенного в однородное гравитационное поле в коническом сосуде высоты h (основание конуса расположено внизу, вверху). Рассмотреть случаи:



\end{task}


\begin{task}

10. Вычислить температурную зависимость теплоемкости двухатомного больцмановского газа, учесть диссоциацию молекул. 


\end{task}


\begin{task}

11. Построить изохоры, изобары и изотермы для бозе-газа. 



\end{task}


\begin{task}

12. Построить изохоры, изобары и изотермы для ферми-газа. 


\end{task}


\begin{task}

13. Вычислить теплоемкость двумерного вырожденного идеального ферми-газа. 


\end{task}


\begin{task}

14. Вычислить теплоемкость черного излучения. 


\end{task}


\begin{task}

Найти равновесную плотность и теплоемкость акустических фононов в кристалле при температурах выше T  и ниже T  дебаевской


\end{task}


\begin{task}

Бспользуя представление оператора смещения гармонического осциллятора $\hat{x}=\left(\frac{\hbar}{2 m \omega}\right)^{1 / 2}\left(\hat{b}^{+}+\hat{b}\right),$ получить формулу $\left\langle e^{i k \hat{x}}\right\rangle=$
$=e^{-\frac{k^{2} \hbar}{4 m \omega}}$ при температуре $T=0$

\end{task}




\begin{task}
17. Описать парамагнетизм Паули и диамагнетизм Ландау. Рассмотреть эффект де Гааза–ван Альфена в двумерном металле. 

\end{task}


\begin{task}

18. Сравнить низкотемпературные зависимости теплоемкости идеальных бозе- и ферми-газов, черного излучения и твердого тела, парамагнетика и ферромагнетика, неидеального бозе-газа и, наконец, сверхпроводника. 


\end{task}






\begin{task}

19. Показать, что фазовая скорость элементарного возбуждения в бозе-конденсате равна гидродинамической скорости звука.


\end{task}



\begin{task}

20

. Найти распределение частиц по импульсам и полное число надконденсатных частиц в идеальном и неидеальном бозе-газах при T = 0 и низких температурах.


\end{task}



\begin{task}

21

Определить свободную энергию одномерной цепочки спинов 1/2 с гамильтонианом

$$\hat{H}=-J \sum_{k}^{N} \hat{\sigma}_{k}^{z} \hat{\sigma}_{k+1}^{z}, \quad \hat{\sigma}_{N+1}^{z}=\hat{\sigma}_{1}^{z}$$
Вычислить теплоёмкость и объяснить причину отсутствия фазового перехода при $ X\ne 0 $
\end{task}



\begin{task}

22. Для ферромагнетика в модели Гейзенберга при T  Tc определить спектр возбуждений (магнонов) и найти температурную зависимость намагниченности и теплоемкости спиновых волн.

\end{task}



\begin{task}

23. Для ферромагнетика в модели Гейзенберга в приближении самосогласованного поля определить температуру Кюри Tc, температурную зависимость магнитной восприимчивости  и спонтанной намагниченности вблизи Tc. Сравнить с результатами теории Ландау.

\end{task}



\begin{task}

Определить корреляционный радиус флуктуации параметра порядка в нулевом внешнем поле вблизи точки фазового перехода II рода. Найти флуктуационную поправку к теплоемкости при T = Tc в теории Гинзбурга–Ландау.

\end{task}





\begin{task}

25. Доказать, что плотность сверхтекучей компоненты электронного газа при T = 0 равна полной плотности числа частиц.  

\end{task}


\begin{task}

26. В модели БКШ определить скачок теплоемкости.

\end{task}

\begin{task}

27. Диагонализуя гамильтониан для фотонов и экситонов с учетом гибридизации, получить спектр поляритонов. 


\end{task}



\begin{task}

28. Мешок Нагаоки (спиновый полярон большого радиуса в антиферромагнетике)


\end{task}




\printindex

\bibliographystyle{plain}
\bibliography{bibliography}


\end{document}
