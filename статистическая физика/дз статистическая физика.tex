\documentclass[a4paper,12pt]{article} % тип документа


\usepackage[T2A]{fontenc} % кодировка
\usepackage[utf8]{inputenc} % кодировка исходного текста
\usepackage[english,russian]{babel} % локализация и переносы

% colors
\usepackage[dvipsnames,table,xcdraw]{xcolor}          
\definecolor{light-blue}{rgb}{0.8,0.85,1}


%symbols
\usepackage{upgreek}
\usepackage{amsmath,amsfonts,amssymb,amsthm,mathtools} %math
% теоремы
\theoremstyle{plain}
\newtheorem{definition}{Определение}[section] 
\newtheorem{theorem}{Теорема}
\newtheorem{example}{Пример}
\numberwithin{equation}{section}



% гиперссылки:
\usepackage{hyperref}
\definecolor{darkblue}{HTML}{0000A0}
\definecolor{linkcolor}{HTML}{0000FF}
\hypersetup{pdfstartview=FitH, citecolor=linkcolor, linkcolor=darkblue,urlcolor=red, colorlinks=true}


% графика:
\usepackage{graphics}
\graphicspath{{pic/}}
\DeclareGraphicsExtensions{.pdf,.png,.jpg, .eps}
\usepackage{caption} % а что без этого летит? (забыл)
\usepackage[section,above,below]{placeins} % управление плавающими объектами (?)
\usepackage{floatflt}
\usepackage{framed}



% работа с таблицами (?)
\usepackage{multirow}
\newcommand{\specialcell}[2][c]{%
	\begin{tabular}[#1]{@{}c@{}}#2\end{tabular}} % перенос строки в ячейке таблицы при пакете multirow


\newcommand{\comment}[1]{} % for multiline comments

% defining red box
\newsavebox{\selvestebox}
\newenvironment{colbox}[1]
{\newcommand\colboxcolor{#1}%
	\begin{lrbox}{\selvestebox}%
		\begin{minipage}{\dimexpr\columnwidth-2\fboxsep\relax}}
		{\end{minipage}\end{lrbox}%
	\begin{center}
		\colorbox[HTML]{\colboxcolor}{\usebox{\selvestebox}}
\end{center}}

% предметный указатель и библиография
\usepackage{makeidx}
\makeindex
\usepackage[nottoc]{tocbibind}

% разметка и стиль (???)
\usepackage[left=2cm, right=2cm, top=2cm, bottom=2cm]{geometry}
\usepackage{fancyhdr}
\pagestyle{fancy}
\fancyhead[L]{\rightmark}
%\lhead{ краткое название}
\chead{}
\rhead{\thepage}
\cfoot{} % get rid of the page number 
\renewcommand{\headrulewidth}{1pt}
\renewcommand{\footrulewidth}{0pt}
\usepackage{indentfirst}

\usepackage{framed}
\usepackage{fancyvrb} %for fraim aroun verbatim
 % основная шапка


\author{Юрий Голубев\\ yura.winter@gmail.com }
\title{статистическая физика}
\date{\today}



\newcommand{\parder}[2]{\frac{\partial {#1}}{\partial {#2}}}

\begin{document}
\maketitle

\begin{abstract}
Задачи по статической физике.
\end{abstract}
%%%%%%%%%%%%%%%%%%%%%%%%%%%%%%%%%%%%%%%%%%%%%%%%%%%%%%%%%%%%%%	
\tableofcontents

\section*{Предисловие}
\addcontentsline{toc}{section}{Предисловие}

Здесь следующие есть интересные вещи:
...




\clearpage
\part{упражнения}


\begin{task}\textbf{1}

$ N $ молекул идеального газа в объеме $ V $. 
Определить вероятность того, что в объеме $v < V$ находится $ n $ молекул.

Получить приближенное выражение, когда  $v \ll V $. 
Найти среднее число частиц $\overline{n}$ в объеме $ v $, его среднюю абсолютную и относительную флуктуации.  

Вероятность попадания ровно одной молекулы в объем $ V$ равна $ p=\frac{v}{V}$. 
Поэтому вероятность попадания ровно $n$ молекул в объем $V$ равна 
\[ p^n (1-p)^{N-n} \]
В объем сосуда могут попасть разные молекулы, всего нужных нам комбинаций:
\[ C_N^n =\frac{N!}{n!(N-n)!}\]
Поэтому итоговая вероятность:
\[ P(n)=\frac{N!}{n!(N-n)!}p^n (1-p)^{N-n} \]

Поищем предел $v \ll V$,  для него можно считать, что $ p\ll 1, n\gg 1$, также можно предположить, что $ np=\lambda$, которое конечно и не слишком мало, не слишком велико.
В таком случае имеется известный предел - распределение становится распределением Пуассона.
\[ P(n)=\frac{\lambda^k}{k!}e^{-\lambda}=
\frac{np^k}{k!}e^{-np}\]

Среднее значение $\overline{n}$ найдется с помощью
\[ \overline{n}=\frac{v}{V}N=Np \]

Флуктуации найдутся, зная дисперсию $ D \hat{n}= E n^2 - (E n)^2 = N (p-p^2)=Npq $:


\[ \frac{\sqrt{D n}}{\overline{n}}=\frac{\sqrt{Npq}}{Np}=\sqrt{\frac{q}{Np}}. \]

Пусть $ v\ll V, \overline{n}\gg 1 $

$$
P_{u}(v)=P(n, \lambda)=\frac{\lambda^{n} e^{\lambda}}{n !}
$$
Используя формулу Стирлинга, получаем:
$$
P_{n}(\sigma) \approx \frac{e^{-\lambda}}{\sqrt{2 \pi n}}\left(\frac{e \lambda}{n}\right)^{n}=\frac{1}{\sqrt{2 \pi n}} 
\operatorname{exp}\left(-\lambda+n+n \ln \left(\frac{\lambda}{n}\right)\right)
$$

Просто преобразуем $ P(v)$, введя $ x\equiv \lambda -n$, тогда логарифм раскрывается так: $\ln \frac{\lambda}{n}=\ln \left(\frac{n+x}{n}\right)=\ln \left(1+\frac{x}{n}\right) \approx \frac{x}{n}-\frac{1}{2}\left(\frac{x}{n}\right)^{2}$
%
$$
P_{4}(v)=\frac{1}{\sqrt{2 \pi n}} \operatorname{exp}\left(-x+x-\frac{1}{2} \frac{x^{2}}{n}\right)=
P_{4}(v)=\frac{1}{\sqrt{2 \pi n}} \operatorname{exp}\left(\frac{(\lambda - n)^2}{2\lambda}\right)
$$
это распределение Гаусса



\end{task}


\begin{task}\textbf{2}

Вычислить $ C_p-C_v$ в  переменных $ V, T $ и $ P, T$. 

Определить $ C_p-C_v$ для больцмановского газа,
газа Ван-дер-Ваальса, ферми и бозе-газа и черного излучения.

Первое начало термодинамики:
$$
\delta Q=d U+p d V,
$$
поэтому:
$$
C_{V}=\left(\frac{\partial U}{\partial T}\right)_{V}
$$
Считая внутреннюю энергию $U$ функцией температуры и объема, можем записать
\begin{equation}\label{1}
C_{p}=C_{V}+\left[\left(\frac{\partial U}{\partial V}\right)_{T}+p\right]\left(\frac{\partial V}{\partial T}\right)_{p}
\end{equation}
Выражение в квадратных скобках в правой части легко вычислить, воспользовавшись фундаментальным равенством Гиббса:
$$
T d S=d U+p d V
$$
Имеем:
$$
T\left(\frac{\partial S}{\partial V}\right)_{T}=\left(\frac{\partial U}{\partial V}\right)_{T}+p
$$
Далее, из соотношения
$$
d F=-S d T-p d V
$$
следует равенство
$$
\left(\frac{\partial S}{\partial V}\right)_{T}=\left(\frac{\partial p}{\partial T}\right)_{V}
$$
Поэтому
$$
\left(\frac{\partial U}{\partial V}\right)_{T}+p=T\left(\frac{\partial p}{\partial T}\right)_{V}
$$
Теперь с помощью соотношения \ref{1} получаем
$$
C_{p}-C_{V}=T\left(\frac{\partial p}{\partial T}\right)_{V}\left(\frac{\partial V}{\partial T}\right)_{P}
$$

И используя 
\[ \left(\frac{\partial z}{\partial x}\right)_{y}\left(\frac{\partial x}{\partial y}\right)_{z}\left(\frac{\partial y}{\partial z}\right)_{x}=-1, \]
запишем:
\begin{equation}\label{2}
C_{p}-C_{V}=
-T\left(\frac{\partial V}{\partial T}\right)_{p}^{2}\left(\frac{\partial V}{\partial p}\right)_{T}^{-1}
\end{equation}
или
\begin{equation}\label{3}
C_{p}-C_{V}=
-T\left(\frac{\partial p}{\partial T}\right)_{V}^{2}\left(\frac{\partial p}{\partial V}\right)_{T}^{-1}
\end{equation}


Теперь применим эти формулы для больцмановского газа, для которого $ pV=\nu RT $, имеем:
\[ \left(\frac{\partial V}{\partial T}\right)_{p}=\frac{\nu R}{p} \]
\[\left(\frac{\partial V}{\partial p}\right)_{T} =-\frac{\nu RT}{p^2} \]
Поэтому при подстановке в \ref{2} много множителей сокращаются и имеем:
\[ C_p-C_V=\nu R \]

Теперь применим эти формулы для газа Ван-дер-Ваальса, для которого $ \left(p+\frac{a\nu^2}{V^{2}}\right)\left(\frac{V}{\nu}-b\right)=R T$. 
Будем для просто ты считать, что $\nu=1$. 
И также тут разумнее подставлять все в \ref{3}
Получаем:
\[ \left(\frac{\partial p}{\partial T}\right)_{V}=\frac{R}{V-b} \]
\[ \left(\frac{\partial p}{\partial V}\right)_{T}=-\frac{RT}{(V-b)^2}+\frac{2a}{V^3} \]
Таким образом, подставляем и получаем:
\[ C_p-C_V=\frac{TR}{T-\frac{2a}{V^3}\left(V-b\right)^2} \]


Теперь применим эти формулы для фермии и бозе-газа, для которых $ \frac{pV}{NT}=1\pm \alpha\left(\frac{T_0}{T}\right)^{3/2}.$ Подставим, посчитаем производные, придем к ответу:
\[ C_p-C_V =\frac{N(\alpha T_0^{3/2}N \mp 2 T^{3/2} V)^2 }{\pm 8 \alpha T_0^{3/2} N T^{3/2}+4 T^3 V^2 } \] 





Теперь применим эти формулы для черного излучения. Для такого: $ p=\frac{a}{3} T^4$. Поэтому для него $ C_p-C_V\rightarrow\infty $
 


\end{task}


\begin{task}\textbf{3}

Вычислить число состояний одноатомного больцмановского газа.

Пусть имеется частиц $N$ частиц в объеме $V$. 
Поступательное движение частиц всегда квазиклассично. 
В классической механике состояние системы характеризуется точкой в $6 N-$ мерном фазовом пространстве
\begin{equation}\label{ss}
\alpha=
\left(\mathbf{r}_{1}, \mathbf{p}_{1}, \mathbf{r}_{2}, \mathbf{p}_{2}, \ldots, \mathbf{r}_{\mathbf{N}}, \mathbf{p}_{\mathbf{N}}\right)
\end{equation}
Число точек в элементе 6 -мерного фазового объема $d r^{3} d p^{3}$ согласно правилу Бора-Зоммерфельда равно отношению этого объема к $(2 \pi \hbar)^{3} .$ 
Обобщение этого правила на случай $N$ одинаковых частиц дает дифференциал числа состояний
$$
d \Gamma_{\alpha}=\frac{1}{N !} \prod_{i=1}^{N} \frac{d^{3} r_{i} d^{3} p_{i}}{(2 \pi \hbar)^{3}}
$$
Произведение дифференциалов поделено на $N !$ для того, чтобы все конфигурации, положения частиц в $6 N$ -мерном фазовом пространстве, 
отличающиеся друг от друга лишь перестановками тождествен ных частиц, учитывались только один раз. 
С помощью \ref{ss} произвольная сумма по состояниям может быть представлена в форме интеграла
$$
\sum_{\alpha} F_{\alpha}=\int d \Gamma_{\alpha} F_{\alpha}
$$


Полное число состоя ний - число точек в $6 N$ -мерном пространстве с энерг ией
$$
E_{\alpha}=\sum_{i=1}^{N} \frac{p_{i}^{2}}{2 m}
$$
А так как $ \Gamma(E)=\sum_{\alpha} \theta\left(E-E_{\alpha}\right)$, то в интервале между 0 и $E$ выражается интегралом:
$$
\Gamma(E)=\int d \Gamma_{\alpha} \theta\left(E-E_{\alpha}\right)=\frac{1}{N !} \int \prod_{i=1}^{N} \frac{d^{3} r_{i} d^{3} p_{i}}{(2 \pi \hbar)^{3}} \theta\left(E-\sum_{i=1}^{N} \frac{p_{i}^{2}}{2 m}\right)
$$
Интегрирование по пространственным координатам каждой частицы дает объем $V,$ и с учетом формулы Стильтьеса получаем
$$
\begin{aligned}
\Gamma(E) &=\left(\frac{V e}{N(2 \pi \hbar)^{3}}\right)^{N} J_{3 N}(E) \\
J_{3 N}(E) &=\int \prod_{i=1}^{N} d^{3} p_{i} \theta\left(E-\sum_{i=1}^{N} \frac{p_{i}^{2}}{2 m}\right) - \text{объем 3n мерного шара радиусом p}
\end{aligned}
$$
%
\[ V_{3N}(p)= \frac{\pi^{3N/2}}{\Gamma (\frac{3N}{2}+1)}p^n \approx \left( \frac{2 e \pi p^2}{3N}\right)^{3N/2}  \]
%
Тогда:
\[ \Gamma(E)= \left( \frac{Ve}{N (2\pi\hbar)^3}\right)^N \left(\frac{4e\pi m E^2}{3N}\right)^{3N/2}\]



\end{task}


\begin{task}\textbf{4}

Вычислить число состояний системы $ N $ независимых спинов 1/2.


Систему $N$ спинов $S=\frac{1}{2},$ находящихся в магнит ном поле $B,$ будем описывать гамильтонианом
$$
H=-2 \mu B \sum_{i=1}^{N}\left(s_{i}^{z}-\frac{1}{2}\right)
$$
Энергия одного спина в магнитном поле, равная $-2 \mu B s^{z}\left(s^{z}=\pm \frac{1}{2}\right),$ сдвинута на константу, чтобы минимальная энергия была равна нулю. 
Если $(\vec{N}-M)$ спинов находятся в основном состоя нии $\left(s^{z}=1 / 2\right),$ а $M$ спинов $-$ в возбужденном $(s^{z}=-1 / 2),$ то система имеет энергию $E=M \Delta E,$ где $\Delta E=2 \mu B .$ 
Такая энергия может быть полу чена числом способов, равным:
\begin{equation}\label{ll}
\Delta \Gamma_{M}=\frac{N !}{M !(N-M) !}
\end{equation}

Это и есть наше требуемое число состояний.

При больших значениях аргумента по формуле Стирлинга факториал приближенно равен
$$
N ! \approx(N / e)^{N}
$$
и выражение \ref{ll} принимает вид
$$
\Delta \Gamma=\left(\frac{N}{e}\right)^{N}\left(\frac{e}{M}\right)^{M}\left(\frac{e}{N-M}\right)^{N-M}=\frac{N^{N}}{M^{M}(N-M)^{N-M}}
$$

Из этого выражения можно найти энтропию и другие характеристики, но о них не спрашивается, так что задача решена.


\end{task}


\begin{task}\textbf{5}

Вычислить число состояний системы $ N $ одинаковых независимых осцилляторов.


За вычетом энергии нулевых колебаний энергия системы равна
$$
E=\Delta E \sum_{i=1}^{N} n_{i}, \quad \Delta E=\hbar \omega
$$
где $n_{i}$ - номер возбуждения $i$ -того осциллятора. Это значение энергии может быть получено числом способов, которое следует из комбинаторики
$$
\Gamma=\frac{(N+M-1) !}{(N-1) ! M !} \approx \frac{\left.(N+M)^{N+M}\right)}{N^{N} M^{M}}
$$




\end{task}


\begin{task}\textbf{6}

Получить выражения для неравновесной энтропии ферми- и бозе-газов



Вероятность произвольного состояния:
$$
w_{\alpha}=w\left(т_{p_{1}}\right) \cdot w\left(u_{p_{2}}\right) w\left(u_{p_{3}}\right)\cdot ...
$$


$$
S=-\sum_{x} w_{\alpha} ln w_{\alpha}=-\sum_{n_{p_1}}\sum_{n_{p_2}}\sum_{n_{p_3}}\ldots 
(w\left(u_{p_{2}}\right) w\left(u_{p_{3}}\right)\cdot ...)
(\ln w(n_{p_1})+\ln w(n_{p_2})+... )
$$

\[ S=-\sum_{n_{p_1}}
w(n_{p_1}) \ln w(n_{p_1}) +\ldots \]



Для ферми газа:$ n_p=\{ 0, 1\} $

\[ -\sum_{n_p}w(n_{p}) \ln w(n_{p})=
-w(n_{p_1}) \ln w(n_{p_1})+
w(n_{p_0}) \ln w(n_{p_0})    \]


среднее число частиц с импульсом $ p: \overline{n_p}=\sum_{n_p}w(n_p) n_p =w (1_p)$
%
\[ w(0_p)=1-\overline{n_p} \]
%
поэтому:
\[ -\sum_{n_p}w(n_p)\ln w(n_p) = -(1-\overline{n_p})\ln (1-\overline{n_p}) - \overline{n_p} \ln \overline{n_p}\]


В итоге энтропия для ферми газа равна:
\[ S_F=-\sum_{p}\Big(
(1-\overline{n_p}) \ln (1-\overline{n_p})+\overline{n_p} \ln \overline{n_p}\Big) \]



Теперь то же для бозе-газа. Для него $ n_p\in \{0, ... \infty\} $


Сумма $ \Big[-\sum_{n_p} w(n_p) \ln (n_p)\Big]$ принимает максимальное значение при заданном $ \overline{n_p}$ при условии на функцию Лагранжа:
\[ L= -\sum_{n_p} w(n_p) \ln (n_p) -\lambda_1 \sum_n n w(n) - \lambda_2 \sum_n w(n)\]

$$
\frac{\partial L}{\partial w(n)}=-\ln w(n)-1 -\lambda_1 n - \lambda_2
$$
$$
w(n)=\exp\left(-1-\lambda_{1} n-\lambda_{2}\right)
$$

$$
\sum w_{n}=1 \Rightarrow \exp \left(-1-\lambda_{1} n-\lambda_{2}\right)=1
$$
$$\sum
\exp \left(-1-\lambda_{2}\right) \sum_{n} \exp (-\lambda_1 n)=1
$$

$$
\sum_{n} \operatorname{exp}\left(-\lambda_{1} n\right)=
\frac{1}{1-e^{-\lambda_{1}}} \Rightarrow \operatorname{exp}\left(-1-\lambda_{2}\right)=1-e^{-\lambda_{1}}
$$


$$
\bar{n}=\sum_{n} w(n) n=\exp \left(-1-\lambda_{2}\right) \sum_{n} n 
\exp \left(-\lambda_{1} n\right)=\exp \left(-1-\lambda_{2}\right)
$$
\[ \left(-\frac{d}{d \lambda_1} \sum_{n} \exp \left(-\lambda_{1} n\right)\right)=-\exp \left(-1-\lambda_{2}\right) \frac{d}{d \lambda_{1}} \frac{1}{1-e^{-\lambda_{1}}} \]
%
\[ \bar{n}=\frac{1}{e^{\lambda_{1}}-1 } \]
\[ e^{-\lambda_{1}}=\frac{\bar{n}}{\bar{n}+1} \]




Поэтому после преобразований получаем:
$$
w(n)=\frac{1}{\bar{n}+1}
\left(\frac{\bar{n}}{\bar{n}+1}\right)^{n}
$$


\begin{multline}
S=-\sum_{n} w(n) \ln w(n)=\sum_{n} w(n)\left(\lambda_{1} n+\lambda_{2}+1\right) =\\
= \sum_{n} w_{n}\left(\lambda_{2}+1\right)+\sum_{n} n w_{n} \lambda_{1}=\left(\lambda_{2}+1\right)+\bar{n} \lambda_{1}=\\
=(\bar{n}+1) \ln (\bar{n}+1)-\bar{n} \ln \bar{n}
\end{multline}

В итоге:
$$
S=\sum_{p}\left[\left(1+\bar{n}_{p}\right) \ln \left(1+\bar{n}_{p}\right)-n_{p} \ln \overline{n_{p}}\right]
$$



\end{task}


\begin{task}\textbf{7}

Вычислить основные термодинамические величины ферми- и бозе-газов при $T=0$

\[ \begin{array}{c}
	N=\sum_{p}\left(N_{p}\right) \approx \frac{2 V}{(2 \pi \hbar)^{3}} \int_{0}^{p_{F}} n_{p} 4 \pi p^{2} d(p)=\frac{V}{3 \pi^{2} \hbar^{3}} p_{F}^{3} \\
	E=\sum_{p}\left(E_{p} n_{p}\right) \approx \frac{2 V 4 \pi}{(2 \pi \hbar)^{3} 2 m} \int_{0}^{p_{F}} p^{4} d(p)=\frac{3}{5} N E_{F} 
\end{array} \]

\[ 	P=-\parder{E}{V}=-\frac{\partial}{\partial(V)}\left(\frac{3}{10}\left(3 \pi^{2}\right)^{\frac{2}{3}} \frac{\hbar^{2}}{m} \frac{N^{\frac{5}{3}}}{V^{\frac{2}{3}}}\right)=\frac{1}{5}\left(3 \pi^{2}\right)^{\frac{2}{3}} \frac{\hbar^{2}}{m} n^{\frac{5}{3}}=\frac{2}{5} n E_{F} \]




\end{task}


\begin{task} \textbf{8}



Из функционала Гинзбурга–Ландау получить выражение для плотности тока в магнитном поле, получить уравнение Лондонов и квантование магнитного потока в сверхпроводящем кольце.























\end{task}





\begin{task} \textbf{9}
	
	
Вычислить среднее от произведения четырех ферми-операторов  
$\left\langle\hat{a}_{k}^{+} \hat{a}_{p}^{+} \hat{a}_{u} \hat{a}_{v}\right\rangle,$ 
где $\langle\cdots\rangle-$ усреднение по состоянию невзаимодействующих частиц с заданной температурой и химпотенциалом.











\end{task}


\begin{task}

10

Записать оператор взаимодействия электронов с внешними электрическим и магнитным полями в представлении вторичного квантования.











\end{task}


\begin{task}

11

Вычислить  $\langle\exp (-i q \hat{x})\rangle, \text { гДе } \hat{x}$ оператор смещения одномерного гармонического осциллятора.




\end{task}


\begin{task}

12

Определить температурную зависимость среднеквадратичного смещения атомов от положения равновесия Rk Rp, где ··· обозначают усреднение по состоянию невзаимодействующих фононов с заданной температурой,  Rk –с мещениеатомав k-направлении. Объяснить происхождение нулевых колебаний.



\end{task}


\begin{task}

13

Используя результаты предыдущей задачи, вычислить среднее от произведения четырех операторов смещения, относяпихся к одной и той же ячейке: $\left\langle\hat{R}_{k} \hat{R}_{p} \hat{R}_{i} \hat{R}_{j}\right\rangle,$ где $\langle\cdots\rangle$ обозначают усреднение по состоянию невзаимодействуюших фононов с заданной температурой, $\hat{R}_{k}-$ смещение атома в $k$ -направлении $(k=x, y, z)$



\end{task}


\begin{task}

14

Для электронов, находящихся под поверхностью Ферми, произвести переход к дырочному представлению. Записать полный гамильтониан идеального ферми-газа, используя операторы рождения и уничтожения квазичастиц (электронов над поверхностью Ферми и дырок под поверхностью Ферми). Определить химический потенциал и энергетический спектр полученных квазичастиц. 



\end{task}


\begin{task}

15

Вычисляя первую поправку термодинамической теории возмущений, найти вклад прямого и обменного взаимодействия для ферми- и бозе-частиц. Сравнить результаты


\end{task}


\begin{task}

16

В преобразовании Боголюбова для электронов получить при T  Tc связь операторов поглощения квазичастиц и поглощения голых электронов.


\end{task}







\clearpage
\part{задачи}



\begin{ttask}\textbf{1}


Показать, что замкнутая система из двух равновесных подсистем имеет максимальную энтропию, 
когда у подсистемы равны температура, давление и химические потенциалы.


Энтропию замкнутой системы, образованной из двух равновесных подсистем, определим как:
$$
S=S_{1}\left(E_{1}\right)+S_{2}\left(E_{2}\right)
$$


Такая энтропия максимальна, когда обе подсистемы имеют одинаковую температуру, химический потенциал и давление.
Действительно:
при постоянстве полной энергии $E=E_{1}+E_{2}$ будет выполняться:
{\Large $$
	\begin{array}{c}
	\frac{d S}{d E_{1}}=\frac{d S_{1}}{d E_{1}}+\frac{d S_{2}}{d E_{2}} \frac{d\left(E-E_{1}\right)}{d E_{1}}=\frac{1}{T_{1}}-\frac{1}{T_{2}}=0 \\ \\
	\frac{d^{2} S}{d E^{2}}=-\left(\frac{1}{T^{2} C_{V}}\right)_{1}-\left(\frac{1}{T^{2} C_{V}}\right)_{2}<0
	\end{array}
	$$}
Поэтому в случае максимума энтропии температуры подсистем одинаковы.

Аналогично, варьируя энтропию системы по объему и числу част иц одной из подсистем при условии постоянства полного объема и полного числа частиц 
$\left(\right.$ Используя $\left.\frac{\partial S}{\partial V}=
\frac{\partial S}{\partial E} \frac{\partial E}{\partial V}=\frac{1}{T}(-p), \frac{\partial S}{\partial N}=
\frac{\partial S}{\partial E} \frac{\partial E}{\partial N}=
\frac{1}{T} \mu\right),$ находим, что Энтропия
максимальна, когда равны друг другу давления и химические потенциалы подсистем.



\end{ttask}


\begin{ttask} \textbf{2}

Найти кривую фазового равновесия газ-жидкость $ p(T) $.


При изменении $T$ и $P$ выполняются равенства
$$
d \mu_{1}=-s_{1} d T^{\prime}+v_{1} d P, \quad d \mu_{2}=-s_{2} d T+v_{2} d P
$$
Поскольку $\mu_{1}(P, T)=\mu_{2}\left(P, T^{\prime}\right),$ то $d \mu_{1}=d \mu_{2},$ отку да следует
$$
\left(s_{2}-s_{1}\right) d T=\left(v_{2}-v_{1}\right) d P \quad \text { или } \quad \frac{d p}{d T}=\frac{s_{2}-s_{1}}{v_{2}-v_{1}}
$$
Введем обозначение: $q_{12}=T\left(s_{2}-s_{1}\right) .$ Тогда последнее уравнение примет вид
$$
\frac{d P}{d T}=
\frac{q_{12}}{T\left(v_{2}-v_{1}\right)}
$$


????



\end{ttask}


\begin{ttask} \textbf{3}

Определить энтропию газа $ N $ невзаимодействующих спинов  $\sigma= 1/2$ в магнитном поле при заданной энергии. 
Определить понятие температуры и показать, что она может быть отрицательной. 
Обсудить температурную зависимость теплоемкости. 
Сравнить с задачей о системе невзаимодействующих двухуровневых частиц.


Вспомним упражнение, в котором мы вычисляли число состояний для системы спинов. Это число оказалось равным
$$
\Delta \Gamma=\left(\frac{N}{e}\right)^{N}\left(\frac{e}{M}\right)^{M}\left(\frac{e}{N-M}\right)^{N-M}=\frac{N^{N}}{M^{M}(N-M)^{N-M}}
$$

Логарифм этой величины можно представить в форме
$$
\sigma^{*}=\ln \Delta \Gamma=-N(n \ln (n)+(1-n) \ln (1-n))
$$
Величина
$$
n=\frac{M}{N}=\frac{E}{(N \Delta E)}
$$

Определим температуру $\tau$  с помощью формул статистической физики:
$$
\begin{array}{l}
\frac{1}{\tau}=\frac{d \sigma}{d E}=\frac{1}{\Delta E} \ln \left(\frac{1-n}{n}\right) \\
\frac{d^{2} \sigma}{d E^{2}}=-\frac{1}{N(\Delta E)^{2} n(1-n)}
\end{array}
$$
Обратим внимание, что с ростом среднего числа возбужденных спинов $n$ от нуля до половины температура $\tau$ растет от нуля до бесконечности, а при дальнейшем возрастании числа $n$ в интервале $1 / 2<n<1$ "температура" $\tau$ отрицательна. 

Состояние с отрицательной температурой возможно только для систем с конечным числом всех состоя ний системы. 
В данном случае это число равно $2^{N}$.


Вроде бы итог этой задачи такой же, как и итоге задачи о двухуровневых частицах.


\end{ttask}


\begin{ttask}\textbf{4}

Определить энтропию газа $ N $ невзаимодействующих осцилляторов при заданной энергии $ E$. 
Получить связь между энергией и температурой $ T$.
Обсудить отличие температурного поведения теплоемкости от предыдущей задачи.

Из упражнения про осциллятор мы имеем значение числа состояний:
$$
\frac{(N+M-1) !}{(N-1) ! M !} \approx \frac{\left.(N+M)^{N+M}\right)}{N^{N} M^{M}}
$$
Статистическая энтропия системы равен логарифму формулы выше:
$$
\sigma=\ln \Delta \Gamma=N(-n \ln (n)+(1+n) \ln (1+n)), \quad n=\frac{M}{N}
$$
здесь $n=M / N=E /(N \Delta E)-$ среднее число возбуждений, приходящихся на Один осциллятор.
Производные статистической энтропии равны
$$
\begin{aligned}
\frac{d \sigma}{d E} &=\frac{1}{\Delta E} \ln \left(\frac{1+n}{n}\right) \\
\frac{d^{2} \sigma}{d E^{2}} &=-\frac{1}{N(\Delta E)^{2} n(1+n)}
\end{aligned}
$$

Число состоя ний системы осцилляторов бесконечно, и состояния с отрицательной температурой отсутствуют.





\end{ttask}


\begin{ttask} \textbf{5}

Вычислить магнитную восприимчивость одноатомного парамагнитного газа $ \chi(T) $ с моментом $ J $.







\end{ttask}


\begin{ttask}\textbf{6}

Вычислить для парамагнитного газа изменение температуры при адиабатическом изменении магнитного поля $(\partial T / \partial H)_{S},$ 
если его свободная энергия может быть представлена в виде: $F=F_{0}(T)-$ $-(1 / 2) \chi(T) H^{2}$




\end{ttask}




\begin{ttask} \textbf{7}
Найти флуктуации 
\[ \overline{\Delta E^{2}}, \overline{\Delta N^{2}}, \overline{\Delta S^{2}}, \overline{\Delta P^{2}}, 
\overline{\Delta S \Delta P}, \overline{\Delta V \Delta P},\overline{\Delta S \Delta T}, 
\overline{\Delta T^{2}}, \overline{\Delta V^{2}}, \overline{\Delta T \Delta V}, \overline{\Delta T \Delta P}, \overline{\Delta S \Delta V} \]

Используя гауссову теорию флуктуаций, выразим в формуле $ w \sim \exp \frac{\Delta p \Delta V-\Delta T \Delta S}{2 k T} $ 
величины $\Delta S$ и $\Delta p$ через флуктуации независимых переменных $V$ и $T:$
\begin{equation}\label{nn}
\begin{aligned}
\Delta S &=\left(\frac{\partial S}{\partial V}\right)_{T} \Delta V+\left(\frac{\partial S}{\partial T}\right)_{V} \Delta T \\
\Delta p &=\left(\frac{\partial p}{\partial V}\right)_{T} \Delta V+\left(\frac{\partial p}{\partial T}\right)_{V} \Delta T
\end{aligned}
\end{equation}С помощью равенства $d F=-S d T-p d V$ имеем $\left(\frac{\partial S}{\partial V}\right)_{T}=$ $=\left(\frac{\partial p}{\partial T}\right)_{V} \cdot$ Далее, $\left(\frac{\partial S}{\partial T}\right)_{V}=\frac{C_{V}}{T} . \quad$ Подставляя $\quad$ эти значения
в \ref{nn} имеем
$$
\Delta S=\left(\frac{\partial p}{\partial T}\right)_{V} \Delta V+\frac{C_{V}}{T} \Delta T
$$
Теперь выражение для плотности вероятности $ w$ после подстановки найденных выражений для $\Delta S$ и $\Delta p$ принимает гауссов вид в переменных $V$ и $T:$
\begin{equation}\label{kkkk}
w \sim \exp \left[-\frac{C_{V}}{2 k T^{2}}(\Delta T)^{2}+\frac{1}{2 k T}\left(\frac{\partial p}{\partial V}\right)_{T}(\Delta V)^{2}\right]
\end{equation}
Из\ref{kkkk} видно, что плотность вероятности распалась на произведение...

Это означает, что флуктуации температуры и объема статистически Независимы:
$$
\langle\Delta V \Delta T\rangle=0
$$

Сравнивая \ref{kkkk} с соотношением $\left\langle x^{2}\right\rangle=\int_{-\infty}^{\infty} x^{2} w(x) d x=\alpha^{-1}$ 
находим
$$
\begin{array}{c}
\left\langle(\Delta T)^{2}\right\rangle=\frac{k T^{2}}{\mathrm{C}_{V}} \\
\left\langle(\Delta V)^{2}\right\rangle=-k T\left(\frac{\partial V}{\partial p}\right)_{T}
\end{array}
$$
Для вычисления средних значений комбинаций, содержащих одну из выбранных независимых переменных, удобно выразить флуктуации второй величины через $\Delta V$ и $\Delta T .$ Тогда получим, например, для $\langle\Delta T \Delta p\rangle$
$$
\langle\Delta T \Delta p\rangle=\left(\frac{\partial p}{\partial V}\right)_{T}\langle\Delta T \Delta V\rangle+\left(\frac{\partial p}{\partial T}\right)_{V}\left\langle(\Delta T)^{2}\right\rangle
$$
Подставляя сюда соотношения $\langle$ найдем
$$
\langle\Delta T \Delta p\rangle=\frac{k T^{2}}{C_{V}}\left(\frac{\partial p}{\partial T}\right)_{V}
$$
Аналогично
$$
\langle\Delta V \Delta p\rangle=\left(\frac{\partial p}{\partial V}\right)_{T}\left\langle(\Delta V)^{2}\right\rangle+\left(\frac{\partial p}{\partial T}\right)_{V}\langle\Delta V \Delta T\rangle
$$
Подставляя $ \left\langle(\Delta T)^{2}\right\rangle $ и $\langle\Delta V \Delta T\rangle=0$ имеем
$$
\langle\Delta V \Delta p\rangle=-k T
$$

Далее,
$$
\begin{aligned}
\langle\Delta S \Delta V\rangle=\left(\frac{\partial S}{\partial V}\right)_{T}\langle&\left.(\Delta V)^{2}\right\rangle+\left(\frac{\partial S}{\partial T}\right)_{V}\langle\Delta V \Delta T\rangle=\\
&=-k T\left(\frac{\partial V}{\partial p}\right)_{T}\left(\frac{\partial p}{\partial T}\right)_{V}=k T\left(\frac{\partial V}{\partial T}\right)_{p}
\end{aligned}
$$
Для вычисления флуктуации $\left\langle(\Delta S)^{2}\right\rangle,\left\langle(\Delta p)^{2}\right\rangle,$ и $\langle\Delta p \Delta S\rangle,$ можно
выразить их через $\Delta V$ и $\Delta T .$ Например,
$$
\left\langle(\Delta S)^{2}\right\rangle=\left\langle\left[\left(\frac{\partial p}{\partial T}\right)_{V} \Delta V+\frac{C_{V}}{T} \Delta T\right]^{2}\right\rangle
$$
Раскрывая квадрат суммы и учитывая формулы других флуктуаций найдем
$$
\left\langle(\Delta S)^{2}\right\rangle=-\left(\frac{\partial p}{\partial T}\right)_{V}^{2}\left(\frac{\partial V}{\partial p}\right)_{T} k T+\frac{C_{V}^{2}}{T^{2}} \frac{k T^{2}}{C_{V}}
$$
Учитывая соотношение
\begin{equation}\label{lll}
C_{p}-C_{V}=-T\left(\frac{\partial p}{\partial T}\right)_{V}^{2}\left(\frac{\partial V}{\partial p}\right)_{T}
\end{equation}

окончательно получаем
$$
\left\langle(\Delta S)^{2}\right\rangle=k C_{p}
$$
Аналогично
\begin{equation}\label{ddd}
\begin{aligned}
\left\langle(\Delta p)^{2}\right\rangle=
\left\langle\left[
\left(\frac{\partial p}{\partial V}\right)_{T} \Delta V+
\left(\frac{\partial p}{\partial T}\right)_{V} \Delta T
\right]^{2}\right\rangle=\\
=-k T&
\left(\frac{\partial p}{\partial V}\right)_{T}+
\left(\frac{\partial p}{\partial T}\right)_{V}^{2} 
\frac{k T^{2}}{C_{V}}
\end{aligned}
\end{equation}
С помощью \ref{lll} имеем
$$
\left(\frac{\partial p}{\partial T}\right)_{V}^{2}=-\frac{C_{p}-C_{V}}{T}\left(\frac{\partial p}{\partial V}\right)_{T}
$$
Подставим это выражение в \ref{ddd} и приведем подобные члены:
$$
\left\langle(\Delta p)^{2}\right\rangle=-k T \frac{\mathrm{C}_{p}}{C_{V}}\left(\frac{\partial p}{\partial V}\right)_{T}=-k T\left(\frac{\partial p}{\partial V}\right)_{S}
$$
Наконец,
$$
\begin{aligned}
\langle\Delta p \Delta S\rangle=\langle[&\left.\left.\left(\frac{\partial p}{\partial V}\right)_{T} \Delta V+\left(\frac{\partial p}{\partial T}\right)_{V} \Delta T\right]\left[\left(\frac{\partial p}{\partial T}\right)_{V} \Delta V+\frac{C_{V}}{T} \Delta V\right]\right\rangle=\\
&=\left(\frac{\partial p}{\partial V}\right)_{T}\left(\frac{\partial p}{\partial T}\right)_{V}\left\langle(\Delta V)^{2}\right\rangle+\frac{C_{V}}{T}\left(\frac{\partial p}{\partial T}\right)_{V}\left\langle(\Delta T)^{2}\right\rangle
\end{aligned}
$$

Подставляя сюда $ \left\langle(\Delta T)^{2}\right\rangle $  и $ \left\langle(\Delta V)^{2}\right\rangle $  приходим к равенству
$$
\langle\Delta p \Delta S\rangle=0
$$



\end{ttask}


\begin{ttask}\textbf{8}

Вычислить для одноатомного и двухатомного больцмановских газов 
$ F, \mu, P, S, C$, $(\partial P)(\partial \rho)_{S}$




\end{ttask}


\begin{ttask}

Найти теплоемкость идеального газа без внутренних степеней свободы, помещенного в однородное гравитационное поле в коническом сосуде высоты h (основание конуса расположено внизу, вверху). Рассмотреть случаи:



\end{ttask}


\begin{ttask}

10. Вычислить температурную зависимость теплоемкости двухатомного больцмановского газа, учесть диссоциацию молекул. 





\end{ttask}


\begin{ttask}

11. Построить изохоры, изобары и изотермы для бозе-газа. 



\end{ttask}


\begin{ttask}

12. Построить изохоры, изобары и изотермы для ферми-газа. 


\end{ttask}


\begin{ttask}

13. Вычислить теплоемкость двумерного вырожденного идеального ферми-газа. 





\end{ttask}


\begin{ttask}14. 

Вычислить теплоемкость черного излучения. 




 

\end{ttask}


\begin{ttask}\textbf{15}

Найти равновесную плотность и теплоемкость акустических фононов в кристалле при температурах выше $ T $  и ниже $ T $  дебаевской




\end{ttask}


\begin{ttask}\textbf{16}

Используя представление оператора смещения гармонического осциллятора 

$\hat{x}=\left(\frac{\hbar}{2 m \omega}\right)^{1 / 2}\left(\hat{b}^{+}+\hat{b}\right),$ 
получить формулу $\left\langle e^{i k \hat{x}}\right\rangle=$
$=e^{-\frac{k^{2} \hbar}{4 m \omega}}$ при температуре $T=0$





\end{ttask}




\begin{ttask} \textbf{17}


Описать парамагнетизм Паули и диамагнетизм Ландау. 
Рассмотреть эффект де Гааза–ван Альфена в двумерном металле. 





\end{ttask}


\begin{ttask}

18. Сравнить низкотемпературные зависимости теплоемкости идеальных бозе- и ферми-газов, черного излучения и твердого тела, парамагнетика и ферромагнетика, неидеального бозе-газа и, наконец, сверхпроводника. 


\end{ttask}






\begin{ttask}

19. Показать, что фазовая скорость элементарного возбуждения в бозе-конденсате равна гидродинамической скорости звука.


\end{ttask}



\begin{ttask}

20

. Найти распределение частиц по импульсам и полное число надконденсатных частиц в идеальном и неидеальном бозе-газах при T = 0 и низких температурах.


\end{ttask}



\begin{ttask}

21

Определить свободную энергию одномерной цепочки спинов 1/2 с гамильтонианом

$$\hat{H}=-J \sum_{k}^{N} \hat{\sigma}_{k}^{z} \hat{\sigma}_{k+1}^{z}, \quad \hat{\sigma}_{N+1}^{z}=\hat{\sigma}_{1}^{z}$$
Вычислить теплоёмкость и объяснить причину отсутствия фазового перехода при $ X\ne 0 $
\end{ttask}



\begin{ttask}

22. Для ферромагнетика в модели Гейзенберга при T  Tc определить спектр возбуждений (магнонов) и найти температурную зависимость намагниченности и теплоемкости спиновых волн.

\end{ttask}



\begin{ttask}

23. Для ферромагнетика в модели Гейзенберга в приближении самосогласованного поля определить температуру Кюри Tc, температурную зависимость магнитной восприимчивости  и спонтанной намагниченности вблизи Tc. Сравнить с результатами теории Ландау.

\end{ttask}



\begin{ttask}

Определить корреляционный радиус флуктуации параметра порядка в нулевом внешнем поле вблизи точки фазового перехода II рода. Найти флуктуационную поправку к теплоемкости при T = Tc в теории Гинзбурга–Ландау.

\end{ttask}





\begin{ttask}

25. Доказать, что плотность сверхтекучей компоненты электронного газа при T = 0 равна полной плотности числа частиц.  

\end{ttask}


\begin{ttask}

26. В модели БКШ определить скачок теплоемкости.

\end{ttask}

\begin{ttask}

27. Диагонализуя гамильтониан для фотонов и экситонов с учетом гибридизации, получить спектр поляритонов. 


\end{ttask}



\begin{ttask}

28. Мешок Нагаоки (спиновый полярон большого радиуса в антиферромагнетике)






\end{ttask}





\end{document}
