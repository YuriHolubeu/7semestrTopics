\documentclass[a4paper,12pt]{article} % тип документа


\usepackage[T2A]{fontenc} % кодировка
\usepackage[utf8]{inputenc} % кодировка исходного текста
\usepackage[english,russian]{babel} % локализация и переносы

% colors
\usepackage[dvipsnames,table,xcdraw]{xcolor}          
\definecolor{light-blue}{rgb}{0.8,0.85,1}


%symbols
\usepackage{upgreek}
\usepackage{amsmath,amsfonts,amssymb,amsthm,mathtools} %math
% теоремы
\theoremstyle{plain}
\newtheorem{definition}{Определение}[section] 
\newtheorem{theorem}{Теорема}
\newtheorem{example}{Пример}
\numberwithin{equation}{section}



% гиперссылки:
\usepackage{hyperref}
\definecolor{darkblue}{HTML}{0000A0}
\definecolor{linkcolor}{HTML}{0000FF}
\hypersetup{pdfstartview=FitH, citecolor=linkcolor, linkcolor=darkblue,urlcolor=red, colorlinks=true}


% графика:
\usepackage{graphics}
\graphicspath{{pic/}}
\DeclareGraphicsExtensions{.pdf,.png,.jpg, .eps}
\usepackage{caption} % а что без этого летит? (забыл)
\usepackage[section,above,below]{placeins} % управление плавающими объектами (?)
\usepackage{floatflt}
\usepackage{framed}



% работа с таблицами (?)
\usepackage{multirow}
\newcommand{\specialcell}[2][c]{%
	\begin{tabular}[#1]{@{}c@{}}#2\end{tabular}} % перенос строки в ячейке таблицы при пакете multirow


\newcommand{\comment}[1]{} % for multiline comments

% defining red box
\newsavebox{\selvestebox}
\newenvironment{colbox}[1]
{\newcommand\colboxcolor{#1}%
	\begin{lrbox}{\selvestebox}%
		\begin{minipage}{\dimexpr\columnwidth-2\fboxsep\relax}}
		{\end{minipage}\end{lrbox}%
	\begin{center}
		\colorbox[HTML]{\colboxcolor}{\usebox{\selvestebox}}
\end{center}}

% предметный указатель и библиография
\usepackage{makeidx}
\makeindex
\usepackage[nottoc]{tocbibind}

% разметка и стиль (???)
\usepackage[left=2cm, right=2cm, top=2cm, bottom=2cm]{geometry}
\usepackage{fancyhdr}
\pagestyle{fancy}
\fancyhead[L]{\rightmark}
%\lhead{ краткое название}
\chead{}
\rhead{\thepage}
\cfoot{} % get rid of the page number 
\renewcommand{\headrulewidth}{1pt}
\renewcommand{\footrulewidth}{0pt}
\usepackage{indentfirst}

\usepackage{framed}
\usepackage{fancyvrb} %for fraim aroun verbatim
 % основная шапка


\author{Юрий Голубев\\ yura.winter@gmail.com }
\title{Задание по квантовой теории поля}
\date{\today}



\newcommand{\parder}[2]{\frac{\partial {#1}}{\partial {#2}}} % в самом деле, это удобно.



\begin{document}
\maketitle

\begin{abstract}
квантовая теория поля
\end{abstract}
%%%%%%%%%%%%%%%%%%%%%%%%%%%%%%%%%%%%%%%%%%%%%%%%%%%%%%%%%%%%%%	
\tableofcontents


\section*{Предисловие}
\addcontentsline{toc}{section}{Предисловие}

тренируемся, практикуемся





\clearpage
\part{Первое задание}


\section{упражнения}


\begin{task}\textbf{1}

Рассмотреть вещественный 4-вектор в представлении группы Лоренца $\left(\frac{1}{2}, \frac{1}{2}\right)$


!!! потом разберу, пока просто из киселева выписки.

4-вектор. Произвольная эрмитово самосопряженная величина $V$ в представлении $\left(\frac{1}{2}, \frac{1}{2}\right),$ т.е. несушая пару индексов $\{\alpha \dot{\alpha}\},$ может быть разложена по базису матриц $\sigma^{n}:$
$$
(\hat{V})_{\alpha \dot{\alpha}}=\sigma_{\alpha \dot{\alpha}}^{n} V_{n}
$$
причем
$$
V^{m}=\frac{1}{2} \operatorname{tr}\left\{\bar{\sigma}^{m} \hat{V}\right\}
$$
так как, очеви Дно,
$$
\operatorname{tr}\left\{\bar{\sigma}^{m} \sigma^{n}\right\}=2 g^{m n}
$$
CorJacно установленным нами законами преобразования верхних и нижних спинорных инДексов, преобразования группы $S L(2, \mathbb{C})$ переводят $\hat{V}$ в эрмитову величину
$$
\hat{V}^{\prime}=\Lambda_{-} \hat{V} \Lambda_{-}^{\dagger}
$$
Которая опять может быть разложена по исходному базису:
$$
V^{\prime} m=\frac{1}{2} \operatorname{tr}\left\{\bar{\sigma}^{m} \hat{V}^{\prime}\right\}
$$
Заметим, что
$$
\operatorname{det} \hat{V}^{\prime}=\operatorname{det}\left\{\Lambda_{-} \hat{V} \Lambda_{-}^{\dagger}\right\}=\left|\operatorname{det} \Lambda_{-}\right|^{2} \operatorname{det} \hat{V}=\operatorname{det} \hat{V}
$$
$\Pi$ ри этом
$$
\hat{V}=\left(\begin{array}{cc}
	V_{0}+V_{3} & V_{1}-i V_{2} \\
	V_{1}+i V_{2} & V_{0}-V_{3}
\end{array}\right)
$$


a paвenctBo детерминантов (3.61) означает, что преобразование сохраняет лоренцИн вариантную длину 4-вектора, т.е. представляет собой элемент группы Лоренца на $4-$ векторах. Это представление является Двузначным, так как матрицы $\Lambda_{-}$ и $-\Lambda_{-}$ приводят К иденти чным преобразованиям 4-вектора. $\mathrm{B}$ случае инфинитезималь ных преобразований $\omega^{k l} \rightarrow 0$
$$
\Lambda_{-}=\mathbb{1}-\frac{i}{2} \sigma_{k l} \omega^{k l}, \quad \Lambda_{-}^{\dagger}=1+\frac{i}{2} \bar{\sigma}_{k l} \omega^{k l}
$$
находим, что
$$
V^{\prime} m=V^{m}+\frac{i}{4} \operatorname{tr}\left\{\bar{\sigma}_{k l} \bar{\sigma}^{m} \sigma^{n}-\sigma_{k l} \sigma^{n} \sigma^{m}\right\} \omega^{k l} V_{n}
$$








\end{task}




\begin{task}\textbf{2}

Доказать равенства
$$
\begin{array}{l}
\left(\sigma^{\mu} \bar{\sigma}^{\nu}+\sigma^{\nu} \bar{\sigma}^{\mu}\right)_{\beta}^{\alpha}=2 g^{\mu \nu} \delta_{\beta}^{\alpha} \\
\left(\bar{\sigma}^{\mu} \sigma^{\nu}+\bar{\sigma}^{\nu} \sigma^{\mu}\right)_{\dot{\beta}}^{\dot{\alpha}}=2 g^{\mu \nu} \delta_{\dot{\beta}}^{\dot{\alpha}}
\end{array}
$$

По определению $ \sigma^{\mu}=(1,\sigma^{i}), \bar{\sigma}^{\mu}= (1,-\sigma^i)$.

Посчитаем отдельно 
$$
\begin{array}{l}
	\left(\sigma^{0} \bar{\sigma}^{i}+\sigma^{i} \bar{\sigma}^{0}\right)=
	(-\sigma^{i}+\sigma^{i})=0
	\\
	\left(\sigma^{i} \bar{\sigma}^{j}+\sigma^{j} \bar{\sigma}^{i}\right)=
-(\sigma^{i}\sigma^{j}+\sigma^{j}\sigma^{i})=-2\delta^{ij}
\end{array}
$$

Поэтому 
\[ \left(\sigma^{\mu} \bar{\sigma}^{\nu}+\sigma^{\nu} \bar{\sigma}^{\mu}\right)_{\beta}^{\alpha}=
2 g^{\mu \nu} \delta_{\beta}^{\alpha} \]


?????????

Что с точечным индексом? в чем его особенность?






\end{task}



\begin{task}\textbf{3}

Доказать равенства
$$
\begin{array}{l}
\operatorname{tr}\left\{\bar{\sigma}_{\lambda \rho} \bar{\sigma}^{\mu \nu}\right\}=
\frac{1}{2}\left\{\delta_{\lambda}^{\mu} \delta_{\rho}^{\nu}-\delta_{\lambda}^{\nu} \delta_{\rho}^{\mu}\right\}
-\frac{\mathrm{i}}{2} \hat{\epsilon}_{\lambda \rho}^{\mu \nu} 
\\
\operatorname{tr}\left\{\sigma_{\lambda \rho} \sigma^{\mu \nu}\right\}=
\frac{1}{2}\left\{\delta_{\lambda}^{\mu} \delta_{\rho}^{\nu}-\delta_{\lambda}^{\nu} \delta_{\rho}^{\mu}\right\}+
\frac{\mathrm{i}}{2} \hat{\epsilon}_{\lambda \rho}^{\mu \nu}
\end{array}
$$

По определению 
$$
\begin{array}{l}
	\sigma^{\mu\nu}=-\sigma^{\nu \mu}=
	\frac{i}{4}(\sigma^\mu \bar{\sigma}^\nu - \sigma^\nu \bar{\sigma}^\mu)
	\\
	\bar{\sigma}^{\mu \nu}=-	\bar{\sigma}^{\nu \mu}=\frac{i}{4}(\bar{\sigma^\mu}\sigma^\nu-\bar{\sigma^\nu}\sigma^\mu)
\end{array}
$$

Посмотрим, как можно их расписать через компоненты.
$$
\begin{array}{l}
	\sigma^{00}=0 
	\\
	\sigma^{0i}=	
	-\sigma^{\nu \mu}=
	\frac{i}{4}(\sigma^0 (-\sigma^i) - \sigma^i \sigma^0)=
	-\frac{i}{2}\sigma^i
	\\
	\bar{\sigma^{0i}}=\frac{i}{2}\sigma^i
	\\
	\bar{\sigma}^{ij}=
	\frac{i}{4}(-\sigma^i\sigma^j-(-\sigma^j)\sigma^i)=
	-\frac{i}{4}\cdot 2 i \varepsilon_{ijk}\sigma^k=\frac{1}{2}\varepsilon_{ijk}\sigma^k
	\\
	\bar{\sigma}^{ij}=
	\frac{1}{2}\varepsilon_{ijk}\sigma^k
\end{array}
$$

В последних равенствах использовалось соотношение на матрицы Паули:
\[ \sigma^i \sigma^j=i\varepsilon_{ijk}\sigma^k+\delta_{ij}\sigma^0 \]

Таким образом, исходное уравнение для пространственных индексов $\operatorname{tr}\left\{\sigma_{\lambda \rho} \sigma^{\mu \nu}\right\}$ можно переписать как:
\[ \operatorname{tr}\left\{\sigma_{i j} \sigma^{kl}\right\}=
\operatorname{tr}\left\{ 	\frac{1}{2}\varepsilon_{ijm}\sigma^m 	\frac{1}{2}\varepsilon_{kln}\sigma^n \right\}=
\frac{1}{4}\varepsilon_{ijm}\varepsilon_{kln}\operatorname{tr}\left\{\sigma_{m} \sigma^{n}\right\} =
\frac{1}{4}\varepsilon_{ijm}\varepsilon_{kln}\operatorname{tr}\left\{2\delta^{mn}\right\} \]
И дальше просто преобразуем до конца:
\[ \operatorname{tr}\left\{\sigma_{ij} \sigma^{kl}\right\}=
\frac{1}{2}\varepsilon_{ijm}\varepsilon_{kln}=1\frac{1}{2}(\delta_{ik}\delta_{jl}-\delta_{il}\delta_{jk})\]

А если есть временной индекс, то 
\[ \operatorname{tr}\left\{\sigma_{ij } \sigma^{k0}\right\}=
\operatorname{tr}\left\{\frac{1}{2}\varepsilon_{ijk}\sigma^n \frac{i}{2}\sigma^k\right\}=
\frac{i}{2}\varepsilon_{ijk}=-\frac{i}{2}\varepsilon_{ijk0}=\frac{i}{2}\varepsilon_{ij}^{k0}\]

В последнем переходе показано как от трехмерного символа Леви-Чевиты перейти к четырехмерному.
Также при подъеме пространственной части метрика домножилась на (-1), а при подъеме временной - на 1.

Осталось разобрать случай 
\[ \operatorname{tr}\left\{\sigma_{i0} \sigma^{k0}\right\}=-\left(\frac{i}{2}\right)^2
\operatorname{tr}\left\{\sigma^i\sigma^k\right\}=
\frac{1}{2}\delta_{ik}\equiv 
\frac{1}{2}[\delta_i^k\delta_0^0-\delta_i^0\delta_0^k]\]

Собирая все вместе, получаем
\[ \operatorname{tr}\left\{\sigma_{\lambda \rho} \sigma^{\mu \nu}\right\}=
\frac{1}{2}\left\{\delta_{\lambda}^{\mu} \delta_{\rho}^{\nu}-\delta_{\lambda}^{\nu} \delta_{\rho}^{\mu}\right\}+
\frac{\mathrm{i}}{2} \hat{\epsilon}_{\lambda \rho}^{\mu \nu}
 \]



Теперь то же самое для $ \operatorname{tr}\left\{\bar{\sigma}_{\lambda \rho} \bar{\sigma}^{\mu \nu}\right\} $


????

действуя аналогично, получаем:

..
\[ \operatorname{tr}\left\{\sigma_{\lambda \rho} \sigma^{\mu \nu}\right\}=
\frac{1}{2}\left\{\delta_{\lambda}^{\mu} \delta_{\rho}^{\nu}-\delta_{\lambda}^{\nu} \delta_{\rho}^{\mu}\right\}+
\frac{\mathrm{i}}{2} \hat{\epsilon}_{\lambda \rho}^{\mu \nu}
 \]


\end{task}



\begin{task}\textbf{4}

Показать, что величины
$$
\theta \sigma^{\mu} \bar{\chi}=
\theta^{\alpha} \sigma_{\alpha \dot{\alpha}}^{\mu} \bar{\chi}^{\dot{\alpha}} 
\quad \text { и } \quad 
\bar{\theta} \bar{\sigma}^{\mu} \chi=
\bar{\theta}_{\dot{\alpha}}\left(\bar{\sigma}^{\mu}\right)^{\dot{\alpha} \alpha} \chi_{\alpha}
$$
ведут себя так же, как 4-векторы.

То есть нужно доказать, что






\end{task}



\begin{task}\textbf{5}

Доказать, что	
$\left(\theta_{\alpha}\right)^{\dagger}=\bar{\theta}_{\dot{\alpha}}$ и 
$\left(\bar{\chi}^{\dot{\alpha}}\right)^{\dagger}=\chi^{\alpha}$.

Напомним, что по определению $ \left(\theta_{\alpha}\right)^{\dagger}=\left(\theta_{\dot{\alpha}}\right)^{*}$, 
то есть мы совершаем комплексное сопряжение над спинором, а также заменяем точечный индекс на неточечный.
(????)

Совершим преобразования поднятия индексов и перехода из сопряженного спинора к обычному над $\bar{\theta}_{\dot{\alpha}}$:
\[ \bar{\theta}_{\dot{\alpha}}=\varepsilon_{\dot{\alpha}\dot{\beta}}\bar{\theta}^{\dot{\beta}}= 
\varepsilon_{\dot{\alpha}\dot{\beta}}[i\sigma_2^{\alpha\dot{\beta}} \theta_\alpha]^{*}=
\left(\begin{array}{cc}
	{0} & {1} \\ {-1} & {0}
\end{array}\right) \cdot i
\left(\begin{array}{cc}
	{0} & {-i} \\ {i} & {0}
\end{array}\right)\theta_\alpha^{*}=
\delta_{\dot{\alpha}}^{\alpha}\theta_\alpha^{*}= (\theta_\alpha)^{\dagger}
\]

Однако то, что компоненты равны не значит, что это один и тот же объект, потому что они могут преобразовываться по-разному.
Поэтому проверим, что они преобразуются одинаково:
\[ (\theta^\prime)^\dagger_\alpha=
((\Lambda_{-})^\beta_\alpha \theta_\beta)^\dagger=
\exp \left(-\frac{i}{2}\sigma^{\mu\nu}\omega_{\mu\nu}\right)^\beta_\alpha \theta_\beta \]

????



Теперь докажем что $\left(\bar{\chi}^{\dot{\alpha}}\right)^{\dagger}=\chi^{\alpha}$









\end{task}



\begin{task}\textbf{5}

Покажите, что представления группы Лоренца со спином $s = 1:
(1, 0)$ и $(0, 1)$ отвечают самодуальным и антисамодуальным 
тензорным полям второго ранга в пространстве-времени Минковского, т.е. при
определении поля, дуального к $ B_{\mu\nu}$, как
$$
\tilde{B}^{\mu \nu}=\frac{1}{2} \hat{\epsilon}^{\mu \nu \mu^{\prime} \nu^{\prime}} B_{\mu^{\prime} \nu^{\prime}}
$$

имеют место соотношения самодуальности и антисамодуальности в
пространстве-времени Минковского:

$$
\tilde{B}^{\mu \nu}=\pm \mathrm{i} B^{\mu \nu}
$$
[Hint: При выводе учесть, что представления (1,0) и $(0,1)-$ это бесследовые матрицы в индексах с точками и без точек. $]$







\end{task}


\begin{task}\textbf{7}

Доказать, что квадрат псевдовектора Паули-Любанского имеет вид
$$
W^{2}=-\frac{1}{2}\left\{p^{2} S^{2}-2 p_{\nu} p^{\mu} S_{\mu \lambda} S^{\nu \lambda}\right\}
$$








\end{task}





\section{задачи}


\begin{ttask}$\mathbf{1.}^{C}$ 

Доказать, что компоненты псевдовектора Паули–Любанского для
безмассовых полей равны
$$
W_{0}=
\hbar \boldsymbol{p} \cdot \boldsymbol{s}, \quad 
W^{\alpha}=
\hbar\left\{p_{0} \boldsymbol{s}^{\alpha} \mp \mathrm{i}(\boldsymbol{p} \times \boldsymbol{s})^{\alpha}\right\}
$$




При этом, конечно, на этих полях (т.е. при действии операторов на поля) проекция спина
на ось импульса или, как говорят, спиральность имеет значения
$$
\frac{\boldsymbol{p} \cdot \boldsymbol{j}^{\pm}}{p_{0}}=\partial_{3}^{\pm}=\pm \lambda_{\pm}
$$
Причем на физических полях $W^{2}=0$ Таким образом, среди безмассовых полей со спином базис составляют так назы ваемые
К иральные поля полуцелого спина и поляризованные поля целого спина:
npaeble поля с положительной киральностью и спиральностью $\mathfrak{s}=\lambda_{+}$
$$
\boldsymbol{J}^{-} \equiv 0 \Rightarrow \mathcal{K}=-\mathrm{i} \boldsymbol{s}, \quad \boldsymbol{J}^{+}=\boldsymbol{s}
$$
левъе поля с отрицательной киральностью и спиральностью $\mathfrak{s}=-\lambda_{-}$
$$
\boldsymbol{J}^{+} \equiv 0 \Rightarrow \mathcal{K}=\mathrm{i} \boldsymbol{s}, \quad \boldsymbol{J}^{-}=\boldsymbol{s}
$$
ванны х полей. С учетом
$$
S_{\beta \gamma}=\hbar \epsilon_{\beta \gamma \rho} s^{\rho}
$$
нулевая компонента
$$
W_{0}=-\frac{1}{2} \hat{\epsilon}^{0 \alpha \beta \gamma} p_{\alpha} S_{\beta \gamma}=\hbar p \cdot s=\pm p_{0} \hbar \lambda_{\pm}
$$

При вычислении пространственной компоненты необходимо использовать то, что
$$
\frac{1}{\hbar} S_{0 \gamma}=\mathcal{K}^{\gamma}=\mp \mathrm{i} s^{\gamma}
$$
откуда
$$
W^{\alpha}=-\frac{1}{2} \hat{\epsilon}^{\alpha 0 \beta \gamma} p_{0} S_{\beta \gamma}-\frac{1}{2} 2 \hat{\epsilon}^{\alpha \beta 0 \gamma} p_{\beta} S_{0 \gamma}=\hbar\left\{p_{0} s^{\alpha} \mp \mathrm{i}(\boldsymbol{p} \times \boldsymbol{s})^{\alpha}\right\}
$$







\end{ttask}



\begin{ttask}$\mathbf{2 .}^{C}$ 

Доказать, что квадрат псевдовектора Паули-Любанского для безмассовых полей имеет вид
$$
W^{2}=-4 p_{0}^{2} \hbar^{2}\left\{
\partial^{+} \cdot \partial^{-}-
\frac{1}{p_{0}^{2}}\left(\boldsymbol{p} \cdot \boldsymbol{\jmath}^{+}\right)\left(\boldsymbol{p} \cdot \boldsymbol{\jmath}^{-}\right)-
\frac{\mathrm{i}}{p_{0}} \boldsymbol{p} \cdot\left(\boldsymbol{J}^{+} \times \boldsymbol{j}^{-}\right)
\right\}$$






Псевдовектор Паули-Любанского определяется как
$$
W^{m}=-\frac{1}{2} \hat{\epsilon}^{m n k l} p_{n} S_{k l}
$$


При $p^{2}=0$
$$
\begin{aligned}
	W^{2} &=
	p_{n} p^{k} S_{k l} S^{n l}=p_{n} p^{k} S_{k 0} S^{n 0}+p_{n} p^{k} S_{k \alpha} S^{n \alpha} \\
	&=-p^{\alpha} p^{\beta} S_{0 \alpha} S^{0 \beta}+p_{0}^{2} S^{0 \alpha} S_{0 \alpha}-p^{\gamma} p^{\beta} S^{\gamma \alpha} S_{\beta \alpha}+p_{0} p^{\beta}\left(S_{\beta \alpha} S^{0 \alpha}-S_{0 \alpha} S^{\beta \alpha}\right) \\
	&=\hbar^{2}\left\{(\boldsymbol{p} \cdot \mathcal{K})^{2}-p_{0}^{2} \mathcal{K}^{2}-(\boldsymbol{p} \times s)^{2}+p_{0} \boldsymbol{p} \cdot\{(\boldsymbol{s} \times \mathcal{K})-(\mathcal{K} \times \boldsymbol{s})\}
	\right\}
\end{aligned}
$$
Раскрывая квадрат векторного произведения, находим
$$
\frac{1}{\hbar^{2}} W^{2}=-p_{0}^{2}\left(s^{2}+\mathcal{K}^{2}\right)+(\boldsymbol{p} \cdot \boldsymbol{s})^{2}+(\boldsymbol{p} \cdot \boldsymbol{\mathcal { K }})^{2}+p_{0} \boldsymbol{p} \cdot\{(\boldsymbol{s} \times \mathcal{K})-(\mathcal{K} \times \boldsymbol{s})\}
$$
Выражая генераторы спина и бустов через эрмитовы векторы $\mathcal{J}^{+}$ и $\mathcal{J}^{-},$ получаем
$$
W^{2}=-4 p_{0}^{2} \hbar^{2}\left\{
\boldsymbol{j}^{+} \cdot \boldsymbol{j}^{-}-
\frac{1}{p_{0}^{2}}\left(\boldsymbol{p} \cdot \boldsymbol{\jmath}^{+}\right)\left(\boldsymbol{p} \cdot \boldsymbol{\jmath}^{-}\right)-
\frac{\mathrm{i}}{p_{0}} \boldsymbol{p} \cdot\left(\boldsymbol{\jmath}^{+} \times \boldsymbol{j}^{-}\right)
\right\}
$$




Выберем ось проектирования спина вдоль пространственной компоненты импульса: $\boldsymbol{p}$ $\left(0,0, p_{0}\right),$ откуда $\boldsymbol{p} \cdot \boldsymbol{J}=p_{0} \mathcal{J}_{3},$ тогда
$$
W^{2}=-4 p_{0}^{2} \hbar^{2}\left\{\partial_{1}^{+} \partial_{1}^{-}+\partial_{2}^{+} \partial_{2}^{-}-\mathrm{i}\left(\partial_{1}^{+} \partial_{2}^{-}-\partial_{2}^{+} \partial_{1}^{-}\right)\right\}
$$
или
$$
W^{2}=-4 p_{0}^{2} \hbar^{2}\left\{
\partial_{1}^{+}+\mathrm{i} \partial_{2}^{+}\right\}\left\{\partial_{1}^{-}-\mathrm{i} \partial_{2}^{-}\right\}=-4 p_{0}^{2} \hbar^{2} \partial_{+}^{+} \partial_{-}^{-}
$$


почти доделал, осталось досмотреть мелочи и все.











\end{ttask}



\begin{ttask}$\mathbf{3 .}^{C}$ 

Найти поток частиц с релятивистской нормировкой состояний
$$
\left\langle\boldsymbol{k} \mid \boldsymbol{k}^{\prime}\right\rangle=
2 \epsilon(\boldsymbol{k})(2 \pi)^{3} \delta\left(\boldsymbol{k}-\boldsymbol{k}^{\prime}\right)
$$






\end{ttask}



\begin{ttask} 4.$^{C}$ 
	
Показать, что для свободного комплексного скалярного поля электрический заряд выражается через 
лоренц-инвариантные амплитуды $a(\boldsymbol{k}) \quad$ и $a_{c}(\boldsymbol{k})$ в виде
$$
Q=\int \mathrm{d}^{3} r j^{0}=
\int \frac{\mathrm{d}^{3} k}{(2 \pi)^{3} 2 k_{0}} e\left\{a^{*}(\boldsymbol{k}) a(\boldsymbol{k})-a_{c}^{*}(\boldsymbol{k}) a_{c}(\boldsymbol{k})\right\}
$$





\end{ttask}



\begin{ttask}\textbf{5}

Для решения в виде плоской монохроматической волны для скалярного поля
$$
\phi \mapsto \frac{1}{\sqrt{2 k_{0}}} \mathrm{e}^{\mp \mathrm{i} k x}
$$
найти, что компоненты тензора энрегии-импульса
$$
T_{0}^{0} \mapsto k_{0}, \quad T_{0}^{\alpha} \mapsto \boldsymbol{k}
$$

Тензор энергии-импульса определяется как 
\[ T^\mu_\nu=\parder{L}{\partial_\mu \varphi}\partial_\nu \varphi+
\parder{L}{\partial_\mu \varphi^{*}}\partial_\nu \varphi^{*}-\delta_\nu^\mu L \]

Поэтому в случае скалярного поля, когда $ L=\partial_\mu \partial^\mu \varphi^{*}-m^2\varphi \varphi^{*}$

$ L=\frac{1}{2} (\partial_\nu \varphi_0)^2- \frac{m^2}{2}\varphi_0^2+\frac{1}{2}(\partial_\mu\varphi_1)^2-\frac{m^2}{2} $


и дальше там мы подставляем и вытаксиваем компоненты ТЭИ.






\end{ttask}



\begin{ttask}\textbf{6}


$$
\phi \mapsto \frac{1}{\sqrt{2 k_{0}}} \mathrm{e}^{\mp \mathrm{i} k x}
$$
Найти, ЧТО КОМПОНенТы ТОКа
$$
j^{0} \mapsto \pm e, \quad j^{\alpha} \mapsto \pm e \boldsymbol{k}
$$


\end{ttask}



\begin{ttask}

$\mathbf{7} .^{C} \quad$ Какой вид имеет тензор энергии-импульса релятивистски инВариантного вакуума?


\end{ttask}



\begin{ttask}\textbf{8}


Для правого вейлевского спинора покажите, что из уравнения движения следует тождество
$$
\frac{1}{\hbar} \boldsymbol{W} \bar{\chi}=\frac{1}{2} \boldsymbol{p} \bar{\chi}
$$






\end{ttask}



\begin{ttask}\textbf{9}


Показать, что если
$$
\boldsymbol{p} \cdot \boldsymbol{\sigma} \bar{\chi}(\boldsymbol{p})=|\boldsymbol{p}| \bar{\chi}(\boldsymbol{p}),
$$

то спинор
$$
\chi_{c p}(-\boldsymbol{p})=-\mathrm{i} \sigma_{2} \bar{\chi}^{*}(\boldsymbol{p})
$$
удовлетворяет уравнению
$$
-\boldsymbol{p} \cdot \boldsymbol{\sigma} \chi_{c p}(-\boldsymbol{p})=|\boldsymbol{p}| \chi_{c p}(-\boldsymbol{p})
$$


\end{ttask}



\begin{ttask} \textbf{10.}
	
	 
Вычислить гамильтониан правого вейлевского спИнора в терминах амплитуд релятивистски нормированных мод. 11.C 
Вычислить заряд правого вейлевского 



\end{ttask}



\begin{ttask}

Вычислить заряд правого вейлевского спинора в терминах амплитуд релятивистски нормированных мод.





\end{ttask}



\begin{ttask} \textbf{12} 

Показать, что проекторы на состояния с заданной проекцией
спина частицы на вектор поляризации имеют вид
$$
P_{\pm}=\frac{1}{2}\left(1+\lambda \gamma_{5} \notin\right)
$$
а для античастиц -
$$
P_{\pm}^{c}=\frac{1}{2}\left(1-\lambda \gamma_{5} \notin\right)
$$
НОГО 4 -имПульсу $p:$
$$
\lambda=\pm 1, \quad \epsilon^{2}=-1, \quad \epsilon \cdot p=0
$$


\end{ttask}



\begin{ttask}


$\mathbf{1 3 .}^{C}$ Вычислить сумму по поляризациям дираковских частиц и античастиц:
$$
\Pi(\boldsymbol{p})=\sum_{\lambda} u_{\lambda}(\boldsymbol{p}) \bar{u}_{\lambda}(\boldsymbol{p})=\not p+m c, \quad \Pi^{c}(\boldsymbol{p})=\sum_{\lambda} v_{\lambda}(\boldsymbol{p}) \bar{v}_{\lambda}(\boldsymbol{p})=\not p-m c
$$


\end{ttask}






\begin{ttask}

14 Вывести уравнения Швингера–Дайсона и графическое представление для двухточечной вершинной функции для биспинора Дирака с юкавским взаимодействием с вещественным скалярным полем. Записать правила Фейнмана.


\end{ttask}



\begin{ttask}

15.C Вывести уравнения Швингера–Дайсона и графическое представление для двухточечной вершинной функции для скалярной электродинамики. Записать правила Фейнмана. 



\end{ttask}



\begin{ttask}

16. $^{C}$ Вывести уравнения IIIBИНГера-Дайсона и графическое представЛение для дВухточечной вершинной фунКции для массивного скалярНОГО ПОЛя с самодействием $\lambda \phi^{4} / 4 ! .$ Записать правила Фейнмана.


\end{ttask}



\begin{ttask}

17. $^{C}$ Доказать, ЧТО число петель $N_{L}$ В ДИаграмме с $N_{V}$ степенями Действия взаимодействия $V,$ Числом связных компонент диаграммы $N_{c}$ и числом внутренних линий $N_{I}$ ОПределяется соотношением
$$
N_{L}=N_{I}+N_{c}-N_{V}
$$
Привести примеры одно- и ДВухпетлевых диаграмм с одно- и ДВухсвяЗНыми компонентами в теории с взаимодействием $V \sim \lambda \phi^{4}$


\end{ttask}



\begin{ttask}


18. $^{C}$ Доказать, что разложение связных Диаграмм по петЛЯм совПадает с разложением по постоянной Планка $\hbar$

\end{ttask}













\clearpage
\part{Второе задание}


\section{упражнения}

\begin{task}

8. $^{C}$ Пользуясь антикоммутатором, ВычислИТь следы произведений Гамма-матриц Дирака:
$$
\begin{array}{l}
\operatorname{tr}\left(\gamma^{\mu} \gamma^{\nu}\right), \quad \operatorname{tr}\left(\gamma_{5} \gamma^{\mu}\right), \quad \operatorname{tr}\left(\gamma_{5} \gamma^{\mu} \gamma^{\nu}\right), \quad \operatorname{tr}\left(\gamma^{\mu} \gamma^{\nu} \gamma^{\mu^{\prime}}\right) \\
\operatorname{tr}\left(\gamma_{5} \gamma^{\mu} \gamma^{\nu} \gamma^{\mu^{\prime}}\right), \quad \operatorname{tr}\left(\gamma^{\mu} \gamma^{\nu} \gamma^{\mu^{\prime}} \gamma^{\nu^{\prime}}\right), \quad \operatorname{tr}\left(\gamma_{5} \gamma^{\mu} \gamma^{\nu} \gamma^{\mu^{\prime}} \gamma^{\nu^{\prime}}\right)
\end{array}
$$


\end{task}



\begin{task}

9. $^{C}$ Доказать, что след нечетного числа гамма-матриц Дирака равен НулЮ, а ДЛя четного $n$ Имеет место соотношение редукции
$$
\operatorname{tr}\left(\gamma^{\mu_{1}} \ldots \gamma^{\mu_{n}}\right)=g^{\mu_{1} \mu_{2}} \operatorname{tr}\left(\gamma^{\mu_{3}} \ldots \gamma^{\mu_{n}}\right)+g^{\mu_{1} \mu_{3}} \operatorname{tr}\left(\gamma^{\mu_{2}} \gamma^{\mu_{4}} \ldots \gamma^{\mu_{n}}\right)+\ldots
$$


\end{task}



\begin{task}

Упростить Выражения
$$
\gamma_{\mu} \not{p} \gamma^{\mu}, \quad \gamma_{\mu} \not{p} \not{k} \gamma^{\mu}
$$


\end{task}



\begin{task}

Рассмотреть тождества Фирца для гамма-матриц Дирака.


\end{task}


\section{задачи}


\begin{ttask}

$\mathbf{1 9 .}^{C} \quad$ В ведушем порядке теории возмушений КВантовой электродиНамики вычислить дифференциальное и полное сечения элеткронПОЗитронной аннигиляции в мюон-антимюон: $e^{+} e^{-} \rightarrow \mu^{+} \mu^{-}$


\end{ttask}



\begin{ttask}

$\mathbf{2 0 .}^{C} \quad$ В ведушем порядке теории возмушений Квантовой электродиНамиКи Вычислить дифференциальное и полное сечения элеткрон-
скалярными частицами: $e^{+} e^{-} \rightarrow \pi^{+} \pi^{-} .$ Сравнить распределение по уГЛам в системе центра масс с распределением в случае образования
$\mathrm{M} \mathrm{HOOHOB}$


\end{ttask}



\begin{ttask}

$\mathbf{2 1 .}^{C}$ В ведушем порядке теории возмушений КВантовой электродинаМИКи Вычислить дифференциальное сечение комптоновского рассеяния фотона на электроне: $\gamma e^{-} \rightarrow \gamma e^{-}$


\end{ttask}



\begin{ttask}

$\mathbf{2 2 .}^{C}$ Вычислить сечение рассеяния электронов на мюонном нейтрино
в модели с четырёхфермионном взаимодействием: $e^{-} \nu_{\mu} \rightarrow \nu_{e} \mu^{-}$


\end{ttask}



\begin{ttask}

23. $^{C}$ Вычислить ширину трёхчастичного распада мюона на электрон и нейтрино: $\mu^{-} \rightarrow e^{-} \bar{\nu}_{e} \nu_{\mu}$


\end{ttask}



\begin{ttask}

$\mathbf{2 4 .}^{C}$ Вычислить время дВухчастичного распада заряженного пиона:
$\pi^{-} \rightarrow \mu^{-} \bar{\nu}_{\mu} .$ Сравнить ширину распада пиона на электрон и мюон.



\end{ttask}



\begin{ttask}

$\mathbf{2 5 .}^{C}$ Вычислить время распада нейтрона: $n \rightarrow p e^{-} \bar{\nu}_{e}$


\end{ttask}



\begin{ttask}

26. $^{*}$ Вычислить пирину дВухчастичного распада $Z$ -бозона на нейтри$\mathrm{HO}: Z \rightarrow \nu \bar{\nu}$


\end{ttask}



\begin{ttask}

27. $^{C}$ В ведушем порядке теории возмушений КХД вычислить сечение
$\bar{q} q \rightarrow \bar{c} c$


\end{ttask}



\begin{ttask}

В ведущем порядке теории возмущений КХД вычислить сечение рождения очарованных кварков в глюон-глюоннном слиянии: g g  cc. Рассмотреть синглетный и октетный по цвету вклады в сечение. 


\end{ttask}




\printindex

\bibliographystyle{plain}
\bibliography{bibliography}


\end{document}
