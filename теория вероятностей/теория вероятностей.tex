\documentclass[a4paper,12pt]{article} % тип документа


\usepackage[T2A]{fontenc} % кодировка
\usepackage[utf8]{inputenc} % кодировка исходного текста
\usepackage[english,russian]{babel} % локализация и переносы

% colors
\usepackage[dvipsnames,table,xcdraw]{xcolor}          
\definecolor{light-blue}{rgb}{0.8,0.85,1}


%symbols
\usepackage{upgreek}
\usepackage{amsmath,amsfonts,amssymb,amsthm,mathtools} %math
% теоремы
\theoremstyle{plain}
\newtheorem{definition}{Определение}[section] 
\newtheorem{theorem}{Теорема}
\newtheorem{example}{Пример}
\numberwithin{equation}{section}



% гиперссылки:
\usepackage{hyperref}
\definecolor{darkblue}{HTML}{0000A0}
\definecolor{linkcolor}{HTML}{0000FF}
\hypersetup{pdfstartview=FitH, citecolor=linkcolor, linkcolor=darkblue,urlcolor=red, colorlinks=true}


% графика:
\usepackage{graphics}
\graphicspath{{pic/}}
\DeclareGraphicsExtensions{.pdf,.png,.jpg, .eps}
\usepackage{caption} % а что без этого летит? (забыл)
\usepackage[section,above,below]{placeins} % управление плавающими объектами (?)
\usepackage{floatflt}
\usepackage{framed}



% работа с таблицами (?)
\usepackage{multirow}
\newcommand{\specialcell}[2][c]{%
	\begin{tabular}[#1]{@{}c@{}}#2\end{tabular}} % перенос строки в ячейке таблицы при пакете multirow


\newcommand{\comment}[1]{} % for multiline comments

% defining red box
\newsavebox{\selvestebox}
\newenvironment{colbox}[1]
{\newcommand\colboxcolor{#1}%
	\begin{lrbox}{\selvestebox}%
		\begin{minipage}{\dimexpr\columnwidth-2\fboxsep\relax}}
		{\end{minipage}\end{lrbox}%
	\begin{center}
		\colorbox[HTML]{\colboxcolor}{\usebox{\selvestebox}}
\end{center}}

% предметный указатель и библиография
\usepackage{makeidx}
\makeindex
\usepackage[nottoc]{tocbibind}

% разметка и стиль (???)
\usepackage[left=2cm, right=2cm, top=2cm, bottom=2cm]{geometry}
\usepackage{fancyhdr}
\pagestyle{fancy}
\fancyhead[L]{\rightmark}
%\lhead{ краткое название}
\chead{}
\rhead{\thepage}
\cfoot{} % get rid of the page number 
\renewcommand{\headrulewidth}{1pt}
\renewcommand{\footrulewidth}{0pt}
\usepackage{indentfirst}

\usepackage{framed}
\usepackage{fancyvrb} %for fraim aroun verbatim
 % основная шапка


\author{Юрий Голубев\\ yura.winter@gmail.com }
\title{Задачи теории вероятностей}
\date{\today}


\begin{document}
\maketitle

\begin{abstract}
Задачи теории вероятностей
\end{abstract}
%%%%%%%%%%%%%%%%%%%%%%%%%%%%%%%%%%%%%%%%%%%%%%%%%%%%%%%%%%%%%%	
\tableofcontents

\section*{Предисловие}
\addcontentsline{toc}{section}{Предисловие}

потом переставлю задачи местами, чтобы схожие были рядом


\clearpage
\part{первое задание}


\section{задачи на аксиомы теории вероятностей}

\begin{example}\textbf{T}


Т.1. Пусть $A, B-$ два события. Найти все события $X$ такие, что
$$
\overline{(X \cup A)} \cup \overline{(X \cup \bar{A})}=B
$$

\end{example}


\begin{example}\textbf{T}

T. 2. $\Pi$ усть $A, B-$ два события. Найти все события $X$ такие, что $A X=A B$.



\end{example}


\begin{example}\textbf{T}

Т.3. Пусть $A_{1}, \ldots, A_{n}-$ события. Покаж ите, что
$$
\mathrm{P}\left(\bigcup_{k=1}^{n} A_{k}\right)+\mathrm{P}\left(\bigcap_{k=1}^{n} \overline{A_{k}}\right)=1
$$


\end{example}




\begin{example}\textbf{T}

Т.4. Пусть $A_{1}, A_{2}, \ldots$ - последовательность событий и $\mathrm{P}\left(A_{n}\right)=1$ для всех $n=1,2, \ldots .$ Покажите, что
$$
\mathrm{P}\left(\bigcap_{n=1}^{\infty} A_{n}\right)=1
$$


\end{example}


\begin{example}\textbf{T}




\end{example}


Т.5. Пусть $A_{1}, A_{2}, \ldots$ - последовательность событий. Покажите, что
$$
P\left(\varliminf_{n \rightarrow \infty} A_{n}\right) \leq \varliminf_{n \rightarrow \infty} \mathrm{P}\left(A_{n}\right) \leq \varlimsup_{n \rightarrow \infty} \mathrm{P}\left(A_{n}\right) \leq \mathrm{P}\left(\varlimsup_{n \rightarrow \infty} A_{n}\right)
$$


\begin{example}\textbf{T}

Т.6. Покажите, что для любых двух событий $A$ и $B$ выполняется неравен-
СТВО
$$
|\mathrm{P}(A B)-\mathrm{P}(A) \mathrm{P}(B)| \leq \frac{1}{4}
$$


\end{example}


\section{текстовые задачи о жизни}

\begin{example}\textbf{T}

T.7. Что вероятнее, получить хотя бы одну единицу при бросании четырех игральных костей или хотя бы одну пару единиц при 24 бросаниях двух
KOCT $\mathrm{e} \ddot{\mathbf{H}} ?$


\end{example}


\begin{example}\textbf{T}

Т.8. Объяснить, почему при подбрасывании трёх игральных костей 11 оч-
Ков выпадают чаще, чем 12 очков.


\end{example}




\begin{example}\textbf{T}


Т.9. Из колоды в 52 карты наудачу берется 6 карт. Какова вероятность того, что среди них будут представительницы всех четырех мастей?

\end{example}


\begin{example}\textbf{T}

T.10. $B$ конвертов разложено по одному письму $n$ адресатам. На каждом конверте наудачу написан один из $n$ адресов. Найти вероятность того,
что хотя бы одно письмо пойдет по назначению.


\end{example}






\begin{example}\textbf{T}

Т.11. У билетной кассы стоит очередь в 100 человек. Половина людей в
Очереди имеет 100-рублевые купюры, а вторая половина - 50-рублевые купюры. Изначально в кассе нет денег и стоимость билета - 50 рублей. Какова вероятность, что никому не придется ждать сдачу?


\end{example}


\begin{example}\textbf{T}

Т.12. Расстояние от пункта А до пункта В автобус проходит за 2 минуты, а пешеход - за 15 минут. Интервал движения автобусов 25 минут. Пешеход в случайный момент времени подходит к пункту А и отправляется в В пешком. Найти вероятность того, что в пути пешехода догонит
Очередной автобус.


\end{example}


\begin{example}\textbf{T}

T.13. На отрезке наудачу выбираются две точки. Какова вероятность того, Что из пол учившихся трех отрезков можно составить треугольник?


\end{example}




\begin{example}\textbf{T}

T.14. На плоскость, разлинованную параллельными линиями, расстояние между которыми $L,$ бросают иглу длины $l \leqslant L .$ Какова вероятность
того, Что игла пересечет линию?


\end{example}


\begin{example}\textbf{T}

T.1 5. На отрезок наудачу последовательно одну за другой бросают три точки. Какова вероятность того, что третья по счету точка попадет между ДВумя первыми?


\end{example}





\begin{example}\textbf{T}

Т.16. Случайный эксперимент заключается в последовательном подбрасывании двух игральных костей. Найти вероятность того, что сумма в 5 ОЧКОВ пОявится раныше, чем сумма в 7 очков.

\end{example}


\begin{example}\textbf{T}

$\mathbf{T . 1 7 .}$ Трое игроков по очереди подбрасывают монету. Выигрывает тот, у кого раньше появится кгербж. Найти вероятности выигрыша каждого
ИГрОКа.



\end{example}


\begin{example}\textbf{T}

Т.18. В ящике находится 10 теннисных мячей, из которых 6 новые. Для первой игры наугад берут два мя ча, которые после игры возвращают в яшик. Для второй игры также наугад берут 2 мяча. Найти вероятность
того, что оба мяча, взятые для второй игры, новые.



\end{example}




\begin{example}\textbf{T}

T.19. По каналу связи может быть передана одна из трех последовательностей букв: $A A A A, B B B B, C C C C,$ причем априорные вероятности равны 0,3,0,4 и 0,3 соответственно. Известно, что действие шумов на приемное устройство уменьшает вероятность правильного приема каждой из переданных букв до $0,6,$ а вероятность приема переданной буквы за две другие увеличивается до 0,2 и $0,2 .$ Предполагается, что буквы искажаются независимо друг от друга. Найти вероятность того, что была передана последовательность $A A A A,$ если на приемном устройстве получено $A C A B$.



\end{example}


\begin{example}\textbf{T}

T.20. Имеется три телефонных автомата, которые принимают специальные жетоны. Один из них никогда не работает, второй работает всегда, a третий работает с вероятностью 1/2. Некто имеет три жетона и пытается выяснить, какой из автоматов исправный (работает всегда). Он
делает попытку на одном из автоматов, которая оказывается неудачной.
Затем переходит к другому автомату, на котором две подряд попытки
Оказываются удачными. Какова вероятность, что этот автомат исправНый?


\end{example}





\begin{example}\textbf{T}

T.21. Подводная лодка атакует корабль, выпуская по нему последовательно и независимо одну от другой $n$ торпед. Каждая торпеда попадает в корабль с вероятностью $p .$ При попадании торпеды с вероятностью $\frac{1}{m}$ затопляется один из $m$ отсеков корабля. Определить вероятность гибели корабля, если для этого необходимо затопление не менее двух отсеков.


\end{example}


\section{распределения и случайные величины}


\begin{example}\textbf{T}

T.22. Пусть $\xi$ и $\eta-$ две случайные величины и $\mathrm{P}(\xi \eta=0)=1$ $\mathrm{P}(\xi=1)=\mathrm{P}(\xi=-1)=\mathrm{P}(\eta=1)=\mathrm{P}(\eta=-1)=\frac{1}{4} .$ Найти совмест-
ное распределение этих случайных величин.


\end{example}


\begin{example}\textbf{T}

T.23. Случайные величины $\xi$ и $\eta$ независимы; $\xi$ имеет плотность распределения $f_{\xi}(x),$ а $\mathrm{P}(\eta=0)=\mathrm{P}(\eta=1)=\mathrm{P}(\eta=-1)=\frac{1}{3} .$ Найти закон
распределения случайной величины $\xi+\eta$


\end{example}




\begin{example}\textbf{T}

T.24. В квадрат $\left\{\left(x_{1}, x_{2}\right): 0 \leqslant x_{i} \leqslant 1 ; i=1,2\right\}$ наудачу брошена точка. Пусть $\xi_{1}, \xi_{2}-$ ее координаты. Найти функцию распределения и плотность случайной величины $\eta=\xi_{1}+\xi_{2}$


\end{example}


\begin{example}\textbf{T}

T.25. Пусть $\xi_{k}, k=1,2,-$ независимые сл учайные величины с распредеЛением Пуассона. Найт и распределение их суммы и условное распределение $\xi_{1},$ если известна сумма $\xi_{1}+\xi_{2}$


\end{example}



\begin{example}

T.23. Случайные величины $\xi$ и $\eta$ независимы; $\xi$ имеет плотность распределения $f_{\xi}(x),$ а $\mathrm{P}(\eta=0)=\mathrm{P}(\eta=1)=\mathrm{P}(\eta=-1)=\frac{1}{3} .$ Найти Закон
распределения случайной величины $\xi+\eta$




\end{example}



\begin{example}

T.24. $\mathrm{B}$ квадрат $\left\{\left(x_{1}, x_{2}\right): 0 \leqslant x_{i} \leqslant 1 ; i=1,2\right\}$ наудачу брошена точка. Пусть $\xi_{1}, \xi_{2}-$ ее координаты. Найти функцию распределения и плотность случайной величины $\eta=\xi_{1}+\xi_{2}$




\end{example}



\begin{example}


T.25. Пусть $\xi_{k}, k=1,2,-$ независимые случайные величины с распределением Пуассона. Найти распределение их суммы и условное распределение $\xi_{1},$ если известна сумма $\xi_{1}+\xi_{2}$



\end{example}





\begin{example}

Т.26. Известно, что случайная величина $\xi$ имеет строго возрастаюшууо непрерывную функцию распределения $F_{\xi}(x) .$ Найти распределение случайной величины $\eta=F_{\xi}(\xi) .$




\end{example}





\begin{example}

T.2 7. Пусть $\xi$ имеет имеет стандартное нормальное распределение. Найти функцию распределения и плотность случайной величины $\xi^{2}$




\end{example}



\begin{example}

T. $28 .$ Пусть $X_{1}, \ldots, X_{n}-$ независимые одинаково распределенные случайные величины с плотностью $f(x) .$ Для каждого элементарного события $\omega \in \Omega$ вектор $\left(X_{1}(\omega), \ldots, X_{n}(\omega)\right)$ преобразуем в упорядоченный $\left(X_{(1)}(\omega), \ldots, X_{(n)}(\omega)\right),$ гДе $X_{(1)}(\omega) \leq \ldots \leq X_{(n)}(\omega) .$ Упорядоченный век-
тор $\left(X_{(1)}, \ldots, X_{(n)}\right)$ в математической статистике называют вариационным рядом, а случайные величины $X_{(k)}, k=1, \ldots, n-$ порядковыми статистиками. Покажите, что плотность совместного распределения по-
рядКОВых статистик определяется равенст ВОМ
$$
f_{X_{(1)}, \ldots, X_{(n)}}\left(x_{1}, \ldots, x_{n}\right)=n ! f\left(x_{1}\right) \cdot \ldots \cdot f\left(x_{n}\right) \mathbb{1}_{\left\{x_{1} \leq \ldots \leq x_{n}\right\}}\left(x_{1}, \ldots, x_{n}\right)
$$




\end{example}



\begin{example}


T. $29 .$ Вдоль дороги, длиной в 1 км расположены случайным образом три человека. Найти вероятность того, что никакие два человека не нахоДятся друг от друга на расстоя нии, меньшем $1 / 4$ км.



\end{example}


\section{Математическое ожкидание и Дисперсия. Ковариация и коэффициент корреляции}


\begin{example}

T.30. В $N$ ячеек случайно в соответствии со статистикой Бозе-Эйнштейна (частицы неразличимы и размещение без ограничений) размещаются $n$ частиц. Пусть $\xi-$ число пустых ячеек. Найти $\mathrm{E} \xi$ и $\mathrm{D} \xi$.




\end{example}







\begin{example}


Т.31. Игральная кость подбрасывается $n$ раз. Пусть $\xi-$ число появлений единицы, а $\eta-$ число появлений шестёрки. Найти коэффициент корре-
Лящии этих сл учайных величин.



\end{example}



\begin{example}


Т.32. Подбрасывают две игральные кости. Пусть $\xi_{1}-$ число очков на первой игральной кости, а $\xi_{2}-$ на второй. Определим $\eta_{1}=\xi_{1}+\xi_{2}$ $\eta_{2}=\xi_{1}-\xi_{2} .$ Найти $\operatorname{cov}\left(\eta_{1}, \eta_{2}\right)$ и выяснить, являются ли $\eta_{1}$ и $\eta_{2}$ незави-
СИМыми.



\end{example}



\begin{example}

T.33. Доказать, что если случайные величины $\xi$ и $\eta$ принимают только
по два Значения каждая, то из некоррелируемости следует их независи-
МОСТь.




\end{example}





\begin{example}


Т.34. А вария происходит в точке $X,$ которая равномерно распределена на дороге длиной $L .$ Во время аварии машина скорой помоши находится в точке $Y$, которая также равномерно распределена на дороге. Считая,
что $X$ и $Y$ независимы, найдите математическое ож идание расстояния между машиной скорой помоши и точкой аварии.



\end{example}







\clearpage
\part{Второе задание}


\section{Неравенства Чебышева и Маркова. Законы больших чисел}

\begin{example}

Т. $\mathbf{1 . ~}$ Известно, что число посетителей некоторого салона в день является случайной величиной со средним значением $50 .$
(a) Оценить вероятность того, что число посетителей в конкретный день превысит 75
(b) При условии, что дисперсия числа посетителей в день равна $25,$ оце-
нить вероятность того, что в конкретный день их число будет между 40 и $60 .$




\end{example}



\begin{example}

T. 2. $\mathrm{C}$ помошью неравенства Чебышева оценить вероятность того, что при 1000 бросаниях монеты число выпадений герба окажется в промежутке [450,550]




\end{example}



\begin{example}

T.3. Вероятность того, что изделие качественное, равна $0,9 .$ Каков должен быть объем партии изделий, чтобы с вероятностью $\geq 0,95$ можно было утверждать, что отклонение (по абсолютной вел ичине) доли качествен-
ных изделий от 0,9 не превысит $0,01 ?$




\end{example}





\begin{example}


T.4. (Одностороннее неравенство Чебышева.) Пусть случайная величина $\xi$ имеет нулевое среднее и дисперсию $\sigma^{2} .$ Показать, что для $\varepsilon>0$ выпол-
няется неравенство
$$
P(\xi \geq \varepsilon) \leq \frac{\sigma^{2}}{\sigma^{2}+\varepsilon^{2}}
$$



\end{example}







\begin{example}


T. 5. $($ Неравенство Иенсена. $)$ Пусть $\varphi:(a, b) \rightarrow \mathbb{R}-$ дважды непрерывно дифференцируемая функция и $\varphi^{\prime \prime}(x) \geq 0$ для всех $x \in(a, b)$ (т. е. $\varphi$ Выпуклая функция, и допускается $a=-\infty$ и $b=\infty$ ). Допустим также, что $\xi-$ слу чайная величина, которая принимает значения из $(a, b)$ и $\mathrm{E} \xi=m, \mathrm{E} \varphi(\xi)$ конеч ны. Показать, что тогда
$$
\mathrm{E} \varphi(\xi) \geq \varphi(\mathrm{E} \xi)=\varphi(m)
$$



\end{example}



\begin{example}



T.6. Пусть $\xi-$ слу чайная вел ичина, имеющая нормал ьное распределение с параметрами $\left(a, \sigma^{2}\right) .$ Показать, что для $\varepsilon>0$ выполняется неравенство
$$
\mathrm{P}(|\xi-a| \geq \varepsilon \sigma) \leq \frac{1}{\varepsilon} \sqrt{\frac{2}{\pi}} e^{-\varepsilon^{2} / 2}
$$

$$\int_{\varepsilon}^{\infty} e^{-x^{2} / 2} d x \leq \frac{1}{\varepsilon} e^{-\varepsilon^{2} / 2}, \quad \varepsilon>0$$

\end{example}



\begin{example}

T.7. Пусть $\xi_{1}, \xi_{2}, \ldots$ - последовательность одинаково распределенных сл учайных величин такая, что $\mathrm{E} \xi_{k}=a, \mathrm{D} \xi_{k}=\sigma^{2}$ и $\operatorname{cov}\left(\xi_{i}, \xi_{j}\right)=(-1)^{i-j} v$
$i \neq j .$ Доказать, что для всякого $\varepsilon>0$ выполняется предельное соотно-
Шение
$$
\lim _{n \rightarrow \infty} \mathrm{P}\left(\left|\frac{1}{n} \sum_{k=1}^{n} \xi_{k}-a\right| \geq \varepsilon\right)=0
$$




\end{example}





\begin{example}


T.8. Пусть $\left\{\xi_{n}\right\}-$ последовательность неотрицательных случайных величин с конечными математическими ожиданиями и $\xi_{n} \stackrel{\text { п.н. }}{\rightarrow} \xi, \mathrm{E} \xi<\infty$ $\mathrm{E} \xi_{n} \rightarrow \mathrm{E} \xi$ при $n \rightarrow \infty .$ Покажите, что тогда $\mathrm{E}\left|\xi_{n}-\xi\right| \rightarrow 0$ при $n \rightarrow \infty$
T. e. $\xi n \stackrel{L}{\rightarrow} \xi$



\end{example}



\section{Meтод характеристических функций. Центральная предельная теорема}



\begin{example}


Т.9. Найти характеристическую функцию распределения Лапласа, которое определяется плотностью
$$
f_{\xi}(x)=\frac{1}{2} e^{-|x|}
$$





\end{example}



\begin{example}

T.10. Найти характеристи ческую функцию нормального распределения с параметрами $\left(a, \sigma^{2}\right)$




\end{example}



\begin{example}

Т.11. Найти распределения, которым соответствуют следуюшие характеристические функции:
$$
h_{1}(t)=\cos t ; \quad h_{2}(t)=\frac{1}{2}+\frac{\cos t}{2}+i \frac{\sin t}{6} ; \quad h_{3}(t)=\frac{1}{2-e^{-i t}}
$$




\end{example}





\begin{example}


T.12. Найти распределение, которому соответствует характеристическая фу нкция $h(t)=e^{-t^{2}} \cos t$



\end{example}







\begin{example}

Т. $13 .$ Является ли функция $h(t)=\cos t^{2}$ характеристической?




\end{example}



\begin{example}

Т.14. Случайная вел ичина $\xi_{\lambda}$ распределена по закону Пуассона с параметpom $\lambda .$ Найти
$$
\lim _{\lambda \rightarrow \infty} \mathrm{P}\left(\frac{\xi_{\lambda}-\lambda}{\sqrt{\lambda}} \leqslant x\right)
$$




\end{example}



\begin{example}

T. 15. Используя резул Бтат предыдушей задачи, найти предел
$$
\lim _{n \rightarrow \infty} e^{-n} \sum_{k=n}^{n} \frac{n^{k}}{k !}
$$




\end{example}





\begin{example}

Т.16. Пусть положительные независимые случайные величины $\xi_{m, n}$ $m=1,2, \ldots, n$ одинаково распределены с плотностью $\alpha_{n} e^{-\alpha_{n} x}, x>0,$ где $\alpha_{n}=\lambda n$ и $\lambda>0 .$ Найти предельное при $n \rightarrow \infty$ распределение случайной величины $\xi_{n}=\sum_{m=1}^{n} \xi_{m, n}$




\end{example}


\section{Элементы теории случайных процессов и математической статистики}




\begin{example}


T. $17 .$ Население региона делится по некоторому социально-экономи ческоМу признаку на три подгруппы. Следуюшее поколение с вероятностями 0,$4 ; 0,6$ и $0,2,$ соответственно, остается в своей подгруппе, а если не
Остается, то с равным и вероятностями переходит в Любую из остальных пОДгрупп. Найти:
a) pacпpeделение населения по данному социально-экономическому признаку в следуюшем поколении, если в настоящем поколении в 1 -ой подгруппе было $20 \%$ населения, во 2 -ой подгруппе $-30 \%,$ и в 3 -ей подгру ппе $-50 \%$
б) предельное распределение по данному признаку, которое не меняется при смене поколений.



\end{example}



\begin{example}

T.18. Пусть матрица вероятностей перехода за один шаг цепи Маркова с
ДВумя состоя ния м и и меет вид
$$
\left(\begin{array}{cc}
1-\alpha & \alpha \\
\beta & 1-\beta
\end{array}\right), \quad 0 \leqslant \alpha, \quad \beta \leqslant 1
$$
Найти вероятности перехода за $n$ шагов и предельные вероятности.




\end{example}



\begin{example}

Т.19. Производящая функция процесса Гальтона-Ватсона имеет вид
$$
f(z)=\frac{1}{m+1-m z}, \quad m>0
$$
Выяснить при каких значениях параметра $m$ процесс является: докритическим, крити ческим, надкритическим. Найти вероятность вырождения процесса в надкритическом случае. Показать, что $n$ -тая итерация производяшей функции может быть представлена в виде

\begin{equation}\begin{array}{l}
\qquad f^{n}(z)=\frac{m^{n}-1-m\left(m^{n-1}-1\right) z}{m^{n+1}-1-m\left(m^{n}-1\right) z} \\
\text { при } m \neq 1 \text { и } \\
\qquad f^{n}(z)=\frac{n-(n-1) z}{n+1-n z}
\end{array}\end{equation}


\end{example}





\begin{example}

$\mathbf{T} \cdot \mathbf{2} \mathbf{0} \cdot$ Пусть $\left(W_{t}, t \geq 0\right)-$ винеровский процесс. Для $u \in \mathbb{R}$ определяется
$$
\tau_{u}=\inf \left\{t: W_{t}=u\right\}
$$
- момент первого достижения уровня $u$ траекторией винеровского процесса. Найти плотность распределения случайной величины $\tau_{u}$




\end{example}






\begin{example}

$$
M_{t}=\max _{0 \leq s \leq t} W_{s}, \quad t>0
$$
где $\left(W_{t}, t \geq 0\right)-$ ви неровский процесс. Найти плотность распределения сл у чайной вел и ч ИНы $M_{t}$




\end{example}



\begin{example}

T.22. Пусть $\mathbb{X}=\left(X_{1}, \ldots, X_{n}\right)-$ выборка из пуассоновского распределения
с параметром $\lambda .$ Найти оценку наибольшего правдоподобия параметра
$\lambda$




\end{example}



\begin{example}

T.23. Пусть $\mathbb{X}=\left(X_{1}, \ldots, X_{n}\right)-$ выборка из нормального распределения $N\left(a, \sigma^{2}\right) .$ Найти оцен ки наибольшего правдоподобия параметров $a$ и $\sigma^{2}$




\end{example}






\printindex

\bibliographystyle{plain}
\bibliography{bibliography}


\end{document}
